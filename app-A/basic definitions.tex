\documentclass[../main]{subfiles}
\begin{document}
\section{Basic Definitions}\label{sec:A.1}
The \defemph{standard $n$-simplex}\index{simplex $\Delta^n$} is the convex set $\Delta^n \subset {\mathbb R}^{n + 1}$ consisting of all $(n + 1)$--tuples $(t_0, \ldots, t_n)$ of real numbers with \[t_i \ge 0, \quad t_0 + t_1 + \ldots + t_n = 1.\] Any continuous map from $\Delta^n$ to a topological space $X$ is called a \defemph{singular $n$--simplex} in $X$. The $i$-th \defemph{face} of a singular $n$-simplex $\sigma : \Delta^n \longrightarrow X$ is the singular $(n - 1)$--simplex \[\sigma \circ \phi_i : \Delta^{n - 1} \longrightarrow X\] where the linear imbedding $\phi_i : \Delta^{n - 1} \longrightarrow \Delta^n$ is defined by \[\phi_i(t_0, \ldots, t_{i - 1}, t_{i + 1}, \ldots, t_n) = (t_0, \ldots, t_{i - 1}, 0, t_{i + 1}, \ldots, t_n).\]

For each $n \ge 0$ the \defemph{singular chain group} $C_n(X; \Lambda)$ with coefficients in a \defemph{commutative} ring $\Lambda$ is the free $\Lambda$--module having one generator $[\sigma]$ for each singular $n$--simplex $\sigma$ in $X$. For $n < 0$, the group $C_n(X; \Lambda)$ is defined to be zero. The \defemphi{boundary homomorphism} \[\partial : C_n(X; \Lambda) \longrightarrow C_{n - 1}(X; \Lambda)\] is defined by \[\partial[\sigma] = [\sigma \circ \phi_0] - [\sigma \circ \phi_1] + \ldots + (-1)^n [\sigma \circ \phi_n].\] 
The identity $\partial \circ \partial = 0$ is easily verified. Hence we can define the $n$--th \defemph{singular homology group}\index{singular homology} $\homology_n(X; \Lambda)$ to be the quotient module $Z_n(X; \Lambda)/B_n(X; \Lambda)$\footnote{Editor's note: The elements of $Z_n(X;\Lambda)$ are called cycles and the elements of $B_n(X;\Lambda)$ are called boundaries. In this language we say homology is cycles modded out by boundaries.}, where $Z_n(X; \Lambda)$ is the kernel of $\partial : C_n(X; \Lambda) \longrightarrow C_{n - 1}(X; \Lambda)$ and $B_n(X; \Lambda)$ is the image of $\partial : C_{n + 1}(X; \Lambda) \longrightarrow C_n(X; \Lambda)$. Here and elsewhere the word ``group'' is used, although ``left $\Lambda$-module'' is really meant.

The \defemph{cochain group} $C^n(X; \Lambda)$ is defined to be the dual module \newline $\Hom_\Lambda(C_n(X; \Lambda), \Lambda)$ consisting of all $\Lambda$--linear maps from $C_n(X; \Lambda)$ to $\Lambda$. The value of a cochain $c$ on a chain $\gamma$ will be denoted by $\ip c \gamma \in \Lambda$. The \defemphi{coboundary} of a cochain $c \in C^n(X; \Lambda)$ is defined to be the cochain $\delta c \in C^{n + 1}(X; \Lambda)$ whose value on each $(n + 1)$--chain $\alpha$ is determined by the identity \[\ip {\delta c} \alpha + (-1)^n \ip c {\partial \alpha} = 0.\]
Thus we obtain corresponding modules \[\homology^n(X; \Lambda) = Z^n(X; \Lambda)/B^n(X; \Lambda) = (\ker \delta)/\delta \, C^{n - 1}(X; \Lambda)\] which are called the \defemph{singular cohomology groups}\index{singular cohomology} of $X$\footnote{Editor's note: The elements of $Z^n(X;\Lambda)$ are called cocycles and the elements of $B^n(X;\Lambda)$ are called coboundaries. In this language cohomology is cocycles modded out by coboundaries.}. 

\begin{remark*}
The choice of sign in this formula is based upon the following convention\index{sign conventions}. Whenever two symbols of dimension $m$ and $n$ are permuted, the sign $(-1)^{mn}$ will be introduced. Here the operators $\partial$ and $\delta$ are considered to have dimension $\pm 1$. Thus our sign conventions are the same as those of \cite{maclane_1975} and \cite{dold1972}, but different from those of \cite{eilenbergsteenrod1952} and \cite{spanier1981}. 
\end{remark*}

In some contexts, notably in obstruction theory, it is important to consider cohomology with coefficients in an arbitrary $\Lambda$--module. However in this appendix we consider only cohomology with coefficients in the ring $\Lambda$ itself. 
\end{document}