%Line 5: Is it \varepsilon or \in
%Line 6: Is there contain a dot below c?
\documentclass[../main]{subfiles}
\begin{document}
\section{Cup Products}\label{sec:A.4}
Given cochains $c \in C^m X$ and $c' \in C^n X$, the product\footnote{Editor's note: In this text, this is what we will always mean by the cup product.}\index{cup product} $cc' = c \smile c' \in C^{m+n}X$ is defined as follows. Let $\sigma : \Delta^{m+n} \varrightarrow{} X$ be a singular simplex. By the \defemph{front $m$-face} of $\sigma$ is meant the composition $\sigma \circ \alpha_m : \Delta^m \varrightarrow{} X$ where \[a_m(t_0, \cdots , t_m) = (t_0, \cdots , t_m, 0, \cdots 0).\] Similarly the \defemph{back $n$-face} of $\sigma$ is the composition $\sigma \circ \beta_n$ where \[\beta_n (t_m, t_{m+1}, \cdots t_{m+n}) = (0, \cdots 0, t_m, t_{m+1}, \cdots t_{m+n}).\] Now define $cc'=c \smile c'$ by the identity \[\langle cc', [\sigma]\rangle = (-1)^{mn} \langle c, [\sigma\circ a_m]\rangle \cdot \langle c', [\sigma \circ \beta_n] \rangle \in \Lambda \text{.}\] The product operation is bilinear and associative, but is not commutative. The constant cocycle $1\in C^0 X$ serves as identity element. The formula \[\delta (cc')=(\delta c)c'+(-1)^m c(\delta c')\] is easily verified. This implies that there is a corresponding product operation $\homology^m X \otimes \homology^n X \varrightarrow{} \homology^{m+n}X$ of cohomology classes. On the cohomology level the product operation does commute, up to sign. (See for example \cite[p. 252]{spanier1981}.) In fact, for $a\in \homology^m X, b\in \homology^n X$, one has $ba = (-1)^{mn} ab$. In dealing with graded groups, this property is called commutativity. Thus we say briefly that the cohomology $\homology^* X = (\homology^0 X, \homology^1 X, \homology^2 X, \cdots)$ is commutative as a \defemph{graded ring}.

Now suppose that one is given a pair of spaces $X \supset A$. If the cochain $c$ belongs to the subset $C^m (X, A) \subset C^m X$ (that is if $c[\sigma]=0$ for every $\sigma:\Delta^m \varrightarrow{} A \subset X$) and if $c'\in C^n X$, then clearly $cc'$ belongs to $C^{m+n} (X, A)$. This gives rise to a product operation \[\homology^m (X, A)\otimes \homology^n X \varrightarrow{} \homology^{m+n}(X, A).\] More generally consider two subsets $A, B\subset X$ which satisfy the following:

\begin{hypo} \label{hypo:A.5}
Both $A$ and $B$ are relatively open when considered as subsets of $A\cup B$.
\end{hypo} 

Then one can define a product operation \[\homology^m (X, A) \otimes \homology^n (X, B) \varrightarrow{} \homology^{m+n} (X, A\cup B)\] as follows.\footnote{The difficulty here is caused by the fact that \[C^i (X, A) \cap C^i (X, B) \neq C^i (X, A\cup B)\] since a singular simplex in $X$ may lie in $A\cup B$ without lying either in $A$ or $B$.} Let \[\xhat{C}^i (X; A, B) \subset C^i X\] denote the intersection of the submodules $C^i (X, A)$ and $C^i (X, B)$ of $C^i X$. Given cochains $c \in C^m (X, A)$ and $c' \in C^n (X, B)$, the product $cc'$ clearly belongs to the intersection \[\xhat{C}^{m+n} (X;A, B) = C^{m+n} (X, A) \cap C^{m+n} (X, B).\] Evidently there is a short exact sequence of cochain complexs \[0 \varrightarrow{} C^* (X, A\cup B) \varrightarrow{} \xhat{C}^* (X;A, B) \varrightarrow{} \xhat{C}^* (A \cup B; A, B) \varrightarrow{} 0,\] But the right hand cochain complex is acyclic\footnote{Editor's note: By acyclic we mean that the corresponding cohomology groups are $0$.}, by \cite[p. 197]{eilenbergsteenrod1952} or \cite[p. 252]{spanier1981}. Hence the inclusion \[C^*(X, A\cup B) \varrightarrow{} \xhat{C}^* (X;A, B)\] induces isomorphisms of cohomology groups. Therefore one obtains a cup product operation with values in the required cohomology group $\homology^{m+n} (X, A\cup B)$.
\end{document}