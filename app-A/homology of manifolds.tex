\documentclass[../main]{subfiles}
\begin{document}
\section{Homology of Manifolds}\;
We will now prove some preliminary results which will be needed to construct the fundamental homology class\index{fundamental class!\indexline homology} of a manifold, and to prove the Poincaré Duality Theorem. (Compare Section \ref{thm:11.10}.) 

Let $M$ be a fixed $n$--dimensional manifold, not necessarily compact. We will first study the groups $\homology_i(M, M - K)$ where $K$ denotes a compact subset of $M$. If $K \subset L \subset M$, then the natural homomorphism \[\homology_i(M, M - L) \longrightarrow \homology_i(M, M - K)\] will be denoted by $\rho_K$. The image $\rho_K(\alpha)$ will be thought of as the ``restriction''\index{restriction} of $\alpha$ to $K$. 

\begin{lemma}
\label{lem:A.7}
The groups $\homology_i(M, M - K)$ are zero for $i > n$. A homology class $\alpha \in \homology_n(M, M - K)$ is zero if and only if the restriction \[\rho_K(\alpha) \in \homology_n(M, M - x)\] is zero for each $x \in K$. 
\end{lemma}

\begin{proof}
The proof will be divided into six steps.

\defemph{Case $1$}. Suppose that $M = {\mathbb R}^n$\index{R,Rn,RA,Rinfty,R0n@$\mathbb{R}, \mathbb{R}^n, \mathbb{R}^A, \mathbb{R}^\infty, \mathbb{R}_0^n$} and that $K$ is a compact convex subset.

Let $x$ be a point of $K$, and let $S$ be a large $(n - 1)$--sphere with center $x$. Then $S$ is a deformation retract of both ${\mathbb R}^n - x$ and of ${\mathbb R}^n - K$. From this one sees that \[\homology_i({\mathbb R}^n, {\mathbb R}^n - K) \varrightarrow{\cong} \homology_i({\mathbb R}^n, {\mathbb R}^n - x)\] for all $i$, which completes the proof in Case 1. 

\defemph{Case $2$}. Suppose that $K = K_1 \cup K_2$ where the lemma is known to be true for $K_1, K_2$, and for $K_1 \cap K_2$.

We will make use of the relative Mayer--Vietoris sequence\index{Mayer-Vietoris sequence} \[\ldots \varrightarrow{} \homology_{i + 1}(M, M - (K_1 \cap K_2)) \varrightarrow{\delta} \homology_i(M, M - K) \varrightarrow{s} \homology_i(M, M - K_1) \oplus \homology_i(M, M - K_2) \varrightarrow{} \ldots\] where the homomorphism $s$ is defined by \[s(\alpha) = \rho_{K_1}(\alpha) \oplus \rho_{K_2}(\alpha).\] (See for example \cite[p. 42]{eilenbergsteenrod1952} or \cite[p. 187]{spanier1981}.) Assuming the existence of such a sequence, the proof in Case 2 can be easily completed. Details will be left to the reader.

Here is a brief construction of the sequence. Let $U_j$ denote the open set $M - K_j$. In analogy with the discussion on Section \ref{sec:A.4}, let ${\xhat C}_i(M; U_1, U_2)$ denote the quotient $C_i M/(C_i U_1 + C_i U_2)$ where $C_i U_1 + C_i U_2 \subset C_i(U_1 \cup U_2)$ denotes the free module generated by all singular $i$--simplexes which lie either in $U_1$ or in $U_2$. Then the natural homomorphism \[{\xhat C}_\bullet(M; U_1, U_2) \longrightarrow C_\bullet(M, U_1 \cup U_2)\] induces isomorphisms of homology groups. (Compare the argument in Section \ref{sec:A.4}.) Now the commutative diagram

\begin{center}
\begin{tikzcd}
                                             & {C_i(M, U_1)} \arrow[rd] &                               \\
{C_i(M, U_1 \cap U_2)} \arrow[ru] \arrow[rd] &                          & {{\xhat C}_i(M; U_1, U_2)} \\
                                             & {C_i(M, U_2)} \arrow[ru] &                              
\end{tikzcd}
\end{center}
gives rise to a short exact sequence \[0 \varrightarrow{} C_i(M, U_1 \cap U_2) \varrightarrow{\text{sum}} C_i(M, U_1) \oplus C_i(M, U_2) \varrightarrow{\text{difference}} {\xhat C}_i (M; U_1, U_2) \varrightarrow{} 0\] The associated long exact sequence of homology groups is the required relative Mayer--Vietoris sequence. 

\defemph{Case $3$}. $K \subset {\mathbb R}^n$ is a finite union $K_1 \cup \ldots \cup K_r$ of compact, convex sets. 

Then the lemma can be proved by induction on $r$, making use of Case $1$ and $2$. 

\defemph{Case $4$}. $K$ is an arbitrary compact subset of ${\mathbb R}^n$. 

Given $\alpha \in \homology_i({\mathbb R}^n, {\mathbb R}^n - K)$ choose a compact neighborhood $N$ of $K$ and a class $\alpha' \in \homology_i({\mathbb R}^n, {\mathbb R}^n - N)$ so that $\rho_K(\alpha') = \alpha$. This is possible since we can choose a chain $\gamma \in C_i {\mathbb R}^n$ whose image modulo ${\mathbb R}^n - K$ is a cycle representing $\alpha$. Then the boundary of $\gamma$ is ``supported'' by a compact set disjoint from $K$. We need only choose $N$ small enough to be disjoint from this support.

Cover $K$ by finitely many closed balls $B_1, \ldots, B_r$ such that $B_i \subset N$ and $B_i \cap K \ne \emptyset$. If $i > n$ then $\rho_{B_1 \cup \ldots \cup B_r} \alpha' = 0$ by Case 3, hence $\alpha = 0$. If $i = n$ and $\rho_x(\alpha) = 0$ for each $x \in K$, then clearly $\rho_x(\alpha') = 0$ for each $x \in B_1 \cup \ldots \cup B_r$. (Compare Case 1.) Hence again $\rho_{B_1 \cup \ldots \cup B_r} (\alpha') = 0$ and therefore $\alpha = 0$. 

\defemph{Case $5$}. $K \subset M$ is small enough so as to have a neighborhood homeomorphic to ${\mathbb R}^n$. 

Since $\homology_\ast(M, M - K) \cong \homology_\ast(U, U - K)$, by excision, the assertion in this case follows from Case 4.

\defemph{Case $6$}. $K \subset M$ is arbitrary.

Then $K = K_1 \cup \ldots \cup K_r$ where each $K_j$ is ``small'' as in Case 5. The proof now proceeds by induction on $r$, using Case $2$. This completes the proof of \ref{lem:A.7}.
\end{proof}
\end{document}