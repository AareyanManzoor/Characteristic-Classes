\documentclass[../main]{subfiles}
\begin{document}
\section{Wu's Formula for Stiefel-Whitney Classes}

Let $\sw_i=\sw_i(\tau_M)$\index{Stiefel-Whitney class $\sw_i$} be the $i$-th Stiefel-Whitney class of the tangent bundle of a smooth manifold $M$, or equivalently the $i$-th Stiefel-Whitney class of the normal bundle of the diagonal in $M\times M$. Applying Thom's formula (p. \pageref{ch08:thoms identity})\index{Steenrod squares}
\[
\steenrod^i(u)=(\pi^\ast \sw_i)\smile u
\]
together with the isomorphism
\[
\homology^\ast(E,E_0)
\cong
\homology^\ast (N_\varepsilon, N_\varepsilon-\Delta(M))
\cong
\homology^\ast (M\times M, M\times M - \Delta(M))
\]
of \ref{cor:11.02}, it follows easily that
\[
\steenrod^i(u') = (\sw_i \times 1) \smile u'.
\]
Therefore, restricting to $\homology^\ast(M\times M)$, we obtain $\steenrod^i(u'')=(\sw_i\times 1)\smile u''$.

We will again make use of the fact that the slant product homomorphism
\[
/\beta: \homology^\ast(X\times Y) \to \homology^\ast(X)
\]
is left $\homology^\ast(X)$-linear for any $\beta\in \homology_\ast(Y)$. In particular, the slant product
\[
((\sw_i\times 1)\smile u'')/\mu
\]
is equal to
\[
\sw_i \smile (u''/\mu) = \sw_i.
\]
(Compare the proof of \ref{thm:11.11}.) Since this is equal to $\steenrod^i(u'')/\mu$, we have the following.

\begin{lemma}
\label{lem:11.13}\index{slant product}
If $M$ is compact and smooth, then the Stiefel-Whitney classes of $\tau_M$ are given by the formula $\sw_i=\steenrod^i(u'')/\mu$.
\end{lemma}

As a corollary, if two manifolds $M_1$ and $M_2$ have the same homotopy type, then their Stiefel-Whitney classes must correspond under the resulting isomorphism $\homology^\ast(M_1)\cong \homology^\ast(M_2)$. This follows since the class $u''$ is determined by \ref{thm:11.11}.

In fact, following Wu Wen-Tsün, one can work out an explicit recipe for computing $\sw_i$, given only the mod 2 cohomology ring $\homology^\ast(M)$ and the action of the Steenrod squares on $\homology^\ast(M)$. Consider the additive homomorphism
\[
x\mapsto \langle \steenrod^k(x),\mu \rangle
\]
from $\homology^{n-k}(M)$ to $\mathbb Z/2$. Using the Duality Theorem\index{Poincar\'e duality} \ref{thm:11.10}, there clearly exists one and only one cohomology class
\[
v_k\in \homology^k(M)
\]
which satisfies the identity
\[
\langle v_k\smile x,\mu \rangle = \langle \steenrod^k(x), \mu\rangle
\]
for every $x$. (In fact, if one considers $M$ as the disjoint union of its connected components, then it is easy to check that $v_k$ satisfies the sharper condition
\[
v_k \smile x = \steenrod^k(x) \in \homology^n(M)
\]
for every $x\in \homology^{n-k}(M)$. Of course the class $v_k$ is zero whenever $k>n-k$). We define the \defemph{total Wu class}
\[
v\in \homology^\Pi(M)
=
\homology^0(M)
\oplus
\homology^1(M)
\oplus
\hdots
\oplus
\homology^n(M)
\]
to be the formal sum
\[
v = 1 + v_1 + \hdots + v_n.
\]
Clearly $v$ satisfies and is characterized by the identity
\[
\langle v \smile x, \mu \rangle
=
\langle \steenrod(x), \mu \rangle,
\]
which holds for every cohomology class $x$. Here $\steenrod$ denotes the total squaring operation $\steenrod^0+\steenrod^1+\steenrod^2+\hdots$.

\begin{theorem}[Wu]\index{Wu's formula}
\label{thm:11.14}
The total Stiefel-Whitney class $\sw$ of $\tau_M$ is equal to $\steenrod(v)$. In other words
\[
\sw_k
=
\sum_{i+j=k} \steenrod^i(v_j).
\]
\end{theorem}
\begin{proof}
Choose a basis $\{b_i\}$ for the mod 2 cohomology $\homology^\ast(M)$ and a dual basis $\{b_i^\#\}$, as in \ref{thm:11.10}. Then for any cohomology class $x$ in $\homology^\Pi(M)$ the identity
\[
x = \sum b_i \langle x\smile b_i^\#, \mu \rangle
\]
is easily verified. Applying this identity to the total Wu class $v$ we obtain
\[
v
=
\sum b_i \langle v\smile b_i^\#, \mu \rangle
=
\sum b_i \langle \steenrod(b_i^\#), \mu \rangle.
\]
Therefore $\steenrod(v)$ is equal to
\[
\sum \steenrod(b_i)  \langle \steenrod(b_i^\#), \mu \rangle
=
\sum (\steenrod(b_i)\times \steenrod(b_i^\#))/\mu
=
\steenrod(u'')/\mu
\]
by \ref{thm:11.11}. Hence $\steenrod(v)=\sw$ as required.
\end{proof}

Here is a concrete application to illustrate Wu's theorem. Let $M$ be a compact manifold whose mod 2 cohomology ring is generated by a single element \newline $a\in \homology^k(M)$, which $k\geq 1$. Thus the cohomology $\homology^\ast(M)$ has basis $\{1, a, a^2, \hdots, a^m\}$ and the dimension of $M$ must be equal to $km$, for some integer $m \geq 1$.

\begin{corollary} \label{cor:11.15}
With $M$ as above, the total Stiefel-Whitney class $\sw(\tau_M)$ is equal to
\[
(1+a)^{m+1}
=
1
+
\binom{m+1}{1}a
+
\hdots
+
\binom{m+1}{m}a^m.
\]
\end{corollary}

As an example, the hypothesis of \ref{cor:11.15} is certainly satisfied for the sphere $S^k$, with $m=1$ and $\sw=(1+a)^2=1$. It is also satisfied for the real projective space $\projective^m=\projective^m(\bR)$\index{projective space!\indexline complex $\projective^n(\bC)$}, with cohomology generator $a$ in dimension $k=1$. (Compare \ref{thm:04.05}.) We will see in \S\ref{ch:14} that it is satisfied for the complex projective space $\projective^m(\bC)$, a $2m$-dimensional manifold with cohomology generator in dimension $k=2$. Similarly, it is satisfied for the quaternion projective $m$-space, a $4m$-dimensional manifold with cohomology generator in dimension $k=4$. (See for example \cite{spanier1981}.) Finally, it is satisfied for the Cayley plane\index{Cayley plane}, a $16$-dimensional manifold with cohomology generator $a\in \homology^8(M)$, and with Stiefel-Whitney class $\sw =(1+a)^3+1+a+a^2$. (See Borel \index{Borel, A.} \cite{borel1950}.)

These are essentially the only examples which exist. For according to Adams\index{Adams, J. F.}\cite{adams1960}, if a space $X$ has mod 2 cohomology generated by $a\in \homology^k(X)$ with $k\geq 1$, and if $a^2 \neq 0$, then $k$ must be either 1, 2, 4 or 8. Furthermore, if $a^3\neq 0$, then by \cite{adem} $k$ must be 1, 2 or 4. Thus the manifolds described above give the only possibly truncated polynomial rings on one generator over $\mathbb Z/2$. (Compare the discussion of related problems on page \pageref{problems page 47}.)

\begin{proof}[Proof of \ref{cor:11.15}]
The action of the Steenrod squares on $\homology^\ast(M)$ is evidently given by
\[
\steenrod(a) = a+a^2,
\]
and hence
\[
\steenrod(a^i) = (a+a^2)^i = a^i (1+a)^i.
\]
It follows that the Kronecker index $\langle \steenrod(a^i), \mu \rangle$ is equal to the binomial coefficient $\binom{i}{m-i}$. Applying the formula
\[
\langle \steenrod(a^i), \mu \rangle
=
\langle v\smile a^i, \mu \rangle,
\]
this implies that the coefficient of $a^{m-i}$ in the total Wu class $v$ must also be equal to $\binom{i}{m-i}$. Hence
\[
v = \sum \binom{i}{m-i} a^{m-i}.
\]
Substituting $j$ for $m-i$, it will be more convenient to write this as $v = \sum \binom{m-j}{j} a^j$. Therefore
\[
\sw = \steenrod(v) = \sum \binom{m-j}{j} \steenrod(a^j).
\]
Since we know how to compute $\steenrod(a^j)$, an explicit computation with binomial coefficients should now complete the argument. For example, if $m=5$, then
\[
v = \sum \binom{5-j}{j} a^j = 1+a^2,
\]
hence
\[
\sw = \steenrod(1+a^2) = 1+a^2+a^4.
\]
In general it is clear that the necessary computation, expressing $\sw$ as a polynomial function of $a$, depends only on $m$, being completely independent of the dimension $k$ of $a$. But this gives us a convenient shortcut. For when $k=1$ we already know that this computation must lead to the formula $\sw=(1+a)^{m+1}$ by Theorem \ref{thm:04.05}. Evidently an identical computation, applied to a generator $a$ of higher dimension, must lead to this same formula.
\end{proof}

%%%%%%% problems:

\begin{problem}
\label{prob:11.A}
Prove Lemma \ref{lem:04.03} (that is, compute the mod 2 cohomology of $\projective^n$) by induction on $n$, using the Duality Theorem \ref{thm:11.10} and the cell structure of \ref{cor:06.05}.
\end{problem}

\begin{problem}
\label{prob:11.B}
\defemph{More Poincaré Duality.} \index{Poincar\'e duality} For $M$ compact, using field coefficients, show that
\[
u''/: \homology_{n-k}(M) \to \homology^k(M)
\]
is an isomorphism. Using the cap product operation of Appendix \ref{app:A}, show that the inverse isomorphism is given by\index{cap product}
\[
\cap \mu : \homology^k(M) \to \homology_{n-k}(M)
\]
multiplied by the sign $(-1)^{kn}$.
\end{problem}

\begin{problem}
\label{prob:11.C}\index{embedding}
Let $M=M^n$ and $A=A^p$ be compact oriented manifolds with smooth embedding $i: M\to A$. Let $k=p-n$. Show that the Poincaré duality isomorphism
\[
\cap \mu_A: \homology^k(A) \to \homology_n(A)
\]
maps the cohomology class $u'|_A$ ``dual'' to $M$ to the homology class $(-1)^{nk} i_\ast (\mu_M)$. (We assume that the normal bundle $\nu^k$ is oriented so that $\tau_M\oplus \nu^k$ is orientation preserving isomorphic to $\tau_A|_M$. The proof makes use of the commutative diagram
\[
\adjustbox{scale=0.8}{
\begin{tikzcd}
	\homology^k(A,A-M)\otimes \homology_p(A) & \homology^k(A,A-M)\otimes \homology_p(A,A-M) & \homology^k(N,N-M)\otimes \homology_p(N,N-M) \\
	\homology^k(A)\otimes \homology_p(A) & \homology_n(A) & \homology_n(N)
	\arrow[from=2-1, to=2-2]
	\arrow[from=2-3, to=2-2]
	\arrow[from=1-1, to=2-1]
	\arrow[from=1-1, to=1-2]
	\arrow[from=1-2, to=2-2]
	\arrow[from=1-3, to=2-3]
	\arrow["\cong"{description}, draw=none, from=1-2, to=1-3]
\end{tikzcd}
}
\]
where $N$ is a tubular neighborhood of $M$ in $A$.)\index{tubular neighborhood}
\end{problem}

\begin{problem}
\label{prob:11.D}
Prove that all Stiefel-Whitney numbers of a 3-manifold are zero.\index{Stiefel-Whitney number}
\end{problem}

\begin{problem}
\label{prob:11.E}
Prove the following version of Wu's formula. Let
\[
\xoverline{\steenrod}: \homology^\Pi(M) \to \homology^\Pi(M)
\]
be the inverse of the ring automorphism $\steenrod$. Show that the dual Stiefel-Whitney classes $\xoverline{\sw}_i(\tau_M)$ are determined by the formula\index{Stiefel-Whitney class $\sw_i$!\indexline dual}
\[
\langle \xoverline{\steenrod}(x), \mu \rangle
=
\langle \xoverline \sw \smile x, \mu \rangle,
\]
which holds for every cohomology class $x$. Show that $\xoverline{\sw}_n=0$. If $n$ is not a power of 2, show that $\xoverline{\sw}_{n-1}=0$.
\end{problem}

\begin{problem}
\label{prob:11.F}
Definining Steenrod operations $\steenrod^i: \homology_k(X) \to \homology_{k-i}(X)$ on mod 2 homology by the identity
\[
\langle x, \steenrod^i(\beta) \rangle
=
\langle \xoverline{\steenrod}^i(x), \beta \rangle,
\]
show that
\[
\steenrod(a\cap \beta) = \steenrod(a) \cap \steenrod(\beta),
\]
and that
\[
\steenrod(u''/\beta) = \steenrod(u'')/\steenrod(\beta).
\]
Prove the formulas $\steenrod(\mu)=\xoverline \sw \cap \mu$ and $\xoverline{\steenrod}(\mu)=v\cap \mu$.
\end{problem}

\end{document}