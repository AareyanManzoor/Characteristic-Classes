\documentclass[../main]{subfiles}
\begin{document}
\section{Poincaré Duality and the Diagonal Class}


Let $M$ be a compact smooth manifold. We will study the cohomology of $M$ with coefficients in a field $\Lambda$, continuing to assume either that $M$ is oriented or that $\Lambda=\mathbb{Z} / 2$. 

\begin{theorem}[Duality Theorem]\index{basis}
\label{thm:11.10}
To each basis $b_1, \dots, b_r$ for $\homology^\ast(M)$ there corresponds a dual basis $b_1^\#, \dots, b_r^\#$ for $\homology^\ast(M)$, satisfying the identity
\[
\ip{b_i\smile b_j^\#}{\mu} =\begin{cases} 1 & i=j,\\0 &i\neq j\end{cases}
\]
\end{theorem}

It follows as a corollary that the rank of the vector space $\homology^k(M)$ is equal to the rank of $\homology^{n-k}(M)$. For if a basis element $b_i$ has dimension $k$ then the dual basis element $b_i^\#$ must have dimension $n-k$. In fact, it follows that the vector space $\homology^k(M)$ is isomorphic to the dual vector space $\Hom_\Lambda(\homology^k(M),\,\Lambda)$, using the correspondence $a\mapsto h_a$ where $h_a(b) = \ip{a\smile b}{\mu}$. (For other formulations of Poincaré duality\index{Poincar\'e duality}, see \ref{prob:11.B} and Appendix \ref{app:A}, as well as \cite{spanier1981}, \cite{dold1972}.)

While proving \ref{thm:11.10}, we will simultaneuously give a precise description of the cohomology class $u^{\prime\prime}\in\homology^n(M\times M)$. 

\begin{theorem}
\label{thm:11.11}
With $\{b_i\}$ and $\{b_i^\#\}$ as above, the diagonal cohomology class $u^{\prime\prime}$ is equal to
\[
\sum^r_{i=1} (-1)^{\dim{b_i}} \, b_i\times b_i^\#.
\]
\end{theorem}

\begin{proof}[Proof of \ref{thm:11.10} and \ref{thm:11.11}]
Using the K\"{u}nneth formula,\index{Künneth theorem}
\[
\homology^\ast(M\times M) \cong \homology^\ast(M) \otimes \homology^\ast(M), 
\]
it follows easily that the diagonal class can be represented by a $r$-fold sum
\[
u^{\prime\prime} = b_1 \times c_1 + \dots + b_r \times c_r,
\]
where $c_1,\dots, c_r$ are certain well-defined cohomology classes in $\homology^\ast(M)$ with
\[
\dim{b_i} + \dim{c_i} = n.
\]
Let us apply the homomorphism $/\mu$ to both sides of the identity
\[
(a\times 1) \smile u'' = (1\times a)\smile u''.
\]
On the left side, using the left linearity of the slant product, we obtain
\[
((a\times 1)\smile u'') / \mu
=
a \smile (u''/\mu)=a.
\]
On the right side, substituting $\sum b_j\times c_j$ for $u''$, we obtain
\[
\sum (-1)^{\dim a \, \dim b_j} (b_j \times (a\smile c_j))/\mu
=
\sum (-1)^{\dim a \, \dim b_j} b_j \langle a\smile c_j, \mu \rangle.
\]
Hence this last expression must be equal to $a$. Substituting $b_i$ for $a$, it follows that the coefficient
\[
(-1)^{\dim a \, \dim b_j} \langle b_i\smile c_j, \mu \rangle
\]
of $b_j$ must be $+1$ for $i=j$, and $0$ for $i\neq j$. Setting $b_i^\# = (-1)^{\dim b_i} c_i$, the conclusions follow easily.
\end{proof}
\end{document}