\documentclass[../main]{subfiles}
\begin{document}
\section{Complex Grassmann Manifolds}

Still continuing our complex analogue of real vector bundle theory, we define the \defemph{complex Grassmann manifold} $\grassmannian_{n}(\bC^{n+k})$ to be the set of all complex $n$-planes through the origin in the complex vector space $\bC^{n+k}$. Just as in the real case, this set has a natural structure as smooth manifold. In fact $\grassmannian_{n}(\bC^{n+k})$ has a natural structure as complex analytic manifold of complex dimension $nk$\index{complex manifold}. Furthermore there is a canonical complex $n$-plane bundle\index{cannonical bundle!\indexline complex $\tautological^n$} which we denote by $\tautological^{n}=\tautological^{n}(\bC^{n+k})$ over $\grassmannian_{n}(\bC^{n+k})$. By definition, the total space of $\tautological^{n}$ consists of all pairs $(X, v)$ where $X$ is a complex $n$-plane through the origin in $\bC^{n+k}$ and $v$ is a vector in $X$.

As an example, let us study the special case $n=1$. The Grassmann manifold $\grassmannian_{1}(\bC^{k+1})$ is also known as the complex projective space $\projective^{k}(\bC)$. We will investigate the cohomology ring $\homology^{\ast}(\projective^k(\bC) ;\, \mathbb{Z})$. (Compare Problem \ref{prob:12.C})\index{projective space!\indexline complex $\projective^n(\bC)$}

Applying the Gysin sequence\index{Gysin sequence} to the canonical line bundle $\tautological^{1}= \tautological^{1}(\bC^{k+1})$ over $\projective^k(\bC)$, and using the fact that $\chernclass_{1}(\tautological^{1})=\eulerclass(\tautological_{\bR}^{1})$, we have
\[
\dots \varrightarrow{} \homology^{i+1}(\total_0) \varrightarrow{} \homology^i(\projective^k(\bC)) \varrightarrow{\smile \chernclass_1} \homology^{i+2}(\projective^k(\bC)) \varrightarrow{\pi_0^\ast} \homology^{i+2}(\total_0) \varrightarrow{} \dots
\]
using integer coefficients. The space $\total_0 = \total_0(\tautological^1(\bC^{k+1}))$ is the set of all pairs 
\[
(\text{line through the origin in }\bC^{k+1}, \,\text{non-zero vector in that line})
\]
This can be identified with $\bC^{k+1}\setminus \{0\}$, and hence has the same homotopy type as the unit sphere $\sphere^{2k+1}$. Thus our Gysin sequence reduces to
\[
0 \longrightarrow \homology^i(\projective^k(\bC)) \varrightarrow{\smile \chernclass_1} \homology^{i+2}(\projective^k(\bC)) \longrightarrow 0
\]
for $0\leq i \leq 2k-2$. Hence
\[
\homology^0(\projective^k(\bC)) \cong \homology^{2}(\projective^k(\bC)) \cong \dots \cong \homology^{2k}(\projective^k(\bC)).
\]
Since $\projective^k(\bC)$ is clearly connected, it follows that each $\homology^{2i}(\projective^k(\bC))$ is infinite cyclic generated $\chernclass_1(\tautological^1)^i$ for $i\le k$. Similarly
\[
\homology^1(\projective^k(\bC)) \cong \homology^3(\projective^k(\bC)) \cong \dots \cong \homology^{2k-1}(\projective^k(\bC))
\]
and using the portion
\[
\dots \longrightarrow \homology^{-1}(\projective^k(\bC)) \longrightarrow \homology^{1}(\projective^k(\bC)) \longrightarrow \homology^{1}(\total_0) \longrightarrow \dots 
\]
of the Gysin sequence, we see that these odd-dimensional groups are all zero. That is:

\begin{theorem}\index{cohomology!\indexline of $\projective^n(\bC)$}
\label{thm:14.04}
The cohomology ring $\homology^\ast(\projective^k(\bC);\, \mathbb{Z})$ is a truncated polynomial ring terminating in dimension $2k$, and generated by the Chern class $\chernclass_1(\tautological^1(\bC^{k+1}))$.
\end{theorem}

Now let us let $k$ tend to infinity. The canonical $n$-plane bundle $\tautological^n(\bC^\infty)$ over $\grassmannian_{n}(\bC^\infty)$ will be denoted briefly by $\tautological^n$. Using \ref{thm:14.04}, it follows that $\homology^\ast(\grassmannian_{1}(\bC^\infty))$ is the polynomial ring generated by $\chernclass_1(\tautological^1)$. More generally we will show the following.

\begin{theorem}\index{cohomology!\indexline of $\grassmannian_n(\bC^\infty)$}
\label{thm:14.05} 
The cohomology ring $\homology^\ast(\grassmannian_n(\bC^\infty);\, \mathbb{Z})$ is the polynomial ring over $\mathbb{Z}$ generated by the Chern classes $\chernclass_1(\tautological^n), \dots, \chernclass_n(\tautological^n)$. There are no polynomial relations between these $n$ generators. 
\end{theorem}
\begin{proof}[Proof by induction on $n$]
We may assume that $n \geq 2$, since the Theorem has already been established for $n=1$. Consider the Gysin sequence
\[
\dots \longrightarrow \homology^i(\grassmannian_{n}) \varrightarrow{\smile \chernclass_n} \homology^{i+2n}(\grassmannian_{n}) \varrightarrow{\pi^\ast_0} \homology^{i+2n}(\total_0) \longrightarrow \homology^{i+1}(\grassmannian_{n}) \longrightarrow \dots
\]
associated with the bundle $\tautological^n$, using integer coefficients. 

We will first show that the cohomology ring $\homology^{\ast}(\total_0)$ can be identified with $\homology^{\ast}(\grassmannian_{n-1})$. In fact a canonical map $f: \total_{0} \varrightarrow{} \grassmannian_{n-1}$ is constructed as follows. By definition, a point $(X, v)$ in $\total_{0}$ consists of an $n$-plane $X$ in $\bC^{\infty}$ together with a non-zero vector $v$ in $X$. Let $f(X, v)=X \cap v^\perp$ be the orthogonal complement of $v$ in $X$, using the standard Hermitian metric\index{Hermitian metric}
\[
\ip{(v_{1}, v_{2}, \dots)}{(w_{1}, w_{2}, \dots)}=\sum v_{j} \xoverline{w}_{j}
\]
on $\bC^{\infty}$. Then $f(X, v)$ is a well defined $(n-1)$-plane in $\bC^{\infty}$.

In order to show that $f$ induces cohomology isomorphisms, it is convenient to pass to the sub-bundle $\tautological^{n}(\mathbb{C}^{N}) \subset \tautological^{n}$, consisting of complex n-planes in $N$-space where $N$ is large but finite. Let $f_{N}: \total_{0}(\tautological^{n}(\bC^{N})) \varrightarrow{} \grassmannian_{n-1}(\bC^{N})$ be the corresponding restriction of $f$. For any $(n-1)$-plane $Y$ in $\grassmannian_{n-1}(\bC^N)$ it is evident that the inverse image
\[
f_{N}^{-1}(Y) \subset \total_{0}(\tautological^{n}(\bC^{N}))
\]
consists of all pairs $(X, v)$ where $v \in \bC^N$ is a non-zero vector perpendicular to $Y$, and where $X=Y+\bC v$ is determined by $v$ and $Y$. \defemph{Thus $f_{N}$ can be identified with the projection map
\[
\total_{0}(\omega^{N-n+1}) \longrightarrow \grassmannian_{n-1}(\bC^{N})
\]
where $\omega^{N-n+1}$ is the complex vector bundle whose fiber, over $Y \in \grassmannian_{n-1}(\bC^{N})$, is the orthogonal complement of $Y$ in $\bC^{N}$.}

Using the Gysin sequence of this new vector bundle, it follows that $f_N$ induces cohomology isomorphisms in dimensions $\leq 2(N-n)$. Therefore, taking the direct limit as $N$ tends to infinity, $f$ induces cohomology isomorphisms in all dimensions.

Thus we can insert $\grassmannian_{n-1}$ in place of $\total_{0}$ in the Gysin sequence, obtaining a new exact sequence of the form
\[
\dots \longrightarrow \homology^i(\grassmannian_{n}) \longrightarrow \homology^{i+2n}(\grassmannian_{n}) \varrightarrow{\lambda} \homology^{i+2n}(\grassmannian_{n-1}) \longrightarrow \homology^{i+1}(\grassmannian_{n}) \longrightarrow \dots
\]
with $\lambda = {f^\ast}^{-1}\pi^\ast_0$.

We must show that this homomorphism $\lambda = {f^\ast}^{-1}\pi^\ast_0$ maps the Chern class $\chernclass_i(\tautological^n)$ to $\chernclass_i(\tautological^{n-1})$. This statement is clear for $i=n$, so we may assume that $i<n$. By the definition of Chern classes, the image $\pi_{0}^\ast \chernclass_{i}(\tautological^{n})$ is equal to $\chernclass_{i}(\tautological_{0}^{n})$. But $f: \total_{0} \longrightarrow \grassmannian_{n-1}$ is clearly covered by a bundle map $\tautological_{0}^{n} \mapsto \tautological^{n-1}$. Therefore $f^\ast \chernclass_i(\tautological^{n-1})=\chernclass_i(\tautological_{0}^{n})$ by \ref{lem:14.02}, and it follows that
\[
\lambda \chernclass_i(\tautological^{n}) = {f^\ast}^{-1} \pi_{0}^{\ast} \chernclass_i(\tautological^{n})
\]
is equal to $\chernclass_i(\tautological^{n-1})$ as asserted.

Now let us apply the induction hypothesis. Since $\homology^\ast(\grassmannian_{n-1})$ is generated by the Chern classes $\chernclass_1(\tautological^{n-1}), \dots, \chernclass_{n-1}(\tautological^{n-1})$, it follows that the homomorphism $\lambda$ is surjective, so our sequence reduces to
\[
0 \longrightarrow \homology^i(\grassmannian_{n}) \varrightarrow{\smile \chernclass_n} \homology^{i+2n}(\grassmannian_{n}) \varrightarrow{\lambda} \homology^{i+2n}(\grassmannian_{n-1}) \longrightarrow 0. 
\]
Using this sequence, we will prove, by a subsidiary induction on $i$, that every element $x$ of $\homology^{i+2n} (\grassmannian_{n})$ can be expressed \defemph{uniquely} as a polynomial in the Chern classes $\chernclass_{1}(\tautological^{n}), \dots, \chernclass_{n}(\tautological^{n})$. Certainly the image $\lambda(x)$ can be expressed uniquely as a polynomial $p(\chernclass_{1}(\tautological^{n-1}), \dots, \chernclass_{n-1}(\tautological^{n-1}))$ by our main induction hypothesis. Therefore the element $x-p(\chernclass_{1}(\tautological^{n}), \dots, \chernclass_{n-1}(\tautological^{n}))$ belongs to the kernel of $\lambda$, and hence can be expressed as a product $y\chernclass_{n}(\tautological^{n})$ for some uniquely determined \newline $y \in \homology^{i}(\grassmannian_{n})$. Now $y$ can be expressed uniquely as a polynomial $q(\chernclass_{1}(\tautological^{n}), \dots, \chernclass_{n}(\tautological^{n}))$ by our subsidiary induction hypothesis, hence
\[
x=p(\chernclass_{1}(\tautological^{n}), \dots, \chernclass_{n-1}(\tautological^{n}))+\chernclass_{n}(\tautological^{n}) q(\chernclass_{1}(\tautological^{n}), \dots, \chernclass_{n}(\tautological^{n})).
\]
The polynomials on the right are unique, since if $x$ were also equal to 
\[
p^{\prime}(\chernclass_{1}(\tautological^{n}), \dots, \chernclass_{n-1}(\tautological^{n}))+\chernclass_{n}(\tautological^{n}) q^{\prime}(\chernclass_{1}(\tautological^{n}), \dots, \chernclass_{n}(\tautological^{n}))
\]
then applying $\lambda$ we would see that $p=p^{\prime}$, and dividing the difference by $\chernclass_{n}(\tautological^{n})$ we would see that $q=q^{\prime}$.
\end{proof}

Just as for real $n$-plane bundles (Theorem \ref{thm:05.06}), we can prove:

\begin{theorem}\index{Grassmannian manifold!\indexline complex}\index{universal bundle!\indexline complex}
\label{thm:14.6}
Every complex $n$-plane bundle over a paracompact base space possesses a bundle map into the canonical complex $n$-plane bundle $\tautological^n = \tautological^n(\bC^\infty)$ over $\grassmannian_{n} = \grassmannian_{n}(\bC^\infty)$.
\end{theorem}
In other words every complex $n$-plane bundle over the paracompact base $\B$ is isomorphic to the induced bundle $f^\ast(\tautological^{n})$ for some $f: \B \varrightarrow{} \grassmannian_n$. In fact, just as in the real case, one can actually prove the sharper statement that \defemph{two induced bundles $f^\ast(\tautological^{n})$ and $g^\ast(\tautological^{n})$ are isomorphic if and only if $f$ is homotopic to $g$}. For this reason the bundle $\tautological^{n}=\tautological^{n}(\bC^{\infty})$ is called the \defemph{universal complex $n$-plane bundle}, and its base space $\grassmannian_{n}(\bC^{\infty})$ is called the \defemph{classifying space} for complex $n$-plane bundles. [The notation $\BU(n)$ is often used in the literature for this classifying space.]\index{BU(n)@$\BU(n)$}\index{classifying space}
\end{document} 