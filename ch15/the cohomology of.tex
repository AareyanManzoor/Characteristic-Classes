\documentclass[../main]{subfiles}
\begin{document}
\section{The Cohomology of the Oriented Grassmann Manifold}
Recall that ${\xtilde \grassmannian}_n = {\xtilde \grassmannian}_n({\mathbb R}^\infty)$ denotes the space of oriented real $n$--planes in ${\mathbb R}^\infty$. (The notation $\BSO(n)$\index{BO(n),BSO(n)@$\BO(n),\BSO(n)$} is often used for this classifying space.) We will study the cohomology of ${\xtilde \grassmannian}_n$ with coefficients in an integral domain $\Lambda$ containing $\tfrac 1 2$. This choice of coefficient domain has the effect of killing $2$--torsion. The ``universal'' example of such a domain $\Lambda$ is the ring $\mathbb Z \left[\tfrac 1 2\right]$. However our arguments will work equally well with coefficients in the field of rational numbers $\mathbb Q$, or in any field of characteristic $\ne 2$. The result will be only slightly more complicated than the cases $\homology^\ast(\grassmannian_n({\mathbb R}^\infty); {\mathbb Z}/2)$, $\homology^\ast({\xtilde \grassmannian}_n; {\mathbb Z}/2)$ and $\homology^\ast(\grassmannian_n({\mathbb C}^\infty); \mathbb Z)$ which we have already computed. 

\begin{theorem}\label{thm:15.9}\index{cohomology!\indexline of $\xtilde{\grassmannian}_n$}\index{Grassmannian manifold!\indexline real}
If $\Lambda$ is an integral domain containing $\tfrac 1 2$, then the cohomology ring $\homology^\ast({\xtilde \grassmannian}_{2m + 1}; \Lambda)$ is a polynomial ring over $\Lambda$ generated by the Pontrjagin classes \[\pontrjaginclass_1({\xtilde \gamma}^{2m + 1}), \ldots, \pontrjaginclass_m({\xtilde \gamma}^{2m + 1}).\] Similarly $\homology^\ast({\xtilde \grassmannian}_{2m}; \Lambda)$ is a polynomial ring over $\Lambda$ generated by the Pontrjagin classes $\pontrjaginclass_1(\gamma^{2m}), \ldots, \pontrjaginclass_{m - 1}(\gamma^{2m})$ and the Euler class $\eulerclass({\xtilde \gamma}^{2m})$.
\end{theorem}

In other words for every value of $n$, even or odd, the ring $\homology^\ast({\xtilde \grassmannian}_n; \Lambda)$ is generated by the characteristic classes $\pontrjaginclass_1, \ldots, \pontrjaginclass_{\lfloor n/2\rfloor}$ and $\eulerclass$. These generators are subject only to the relations:
\begin{enumerate}
    \item[] $\eulerclass = 0$ for $n$ odd,
    \item[] $\eulerclass^2 = \pontrjaginclass_{n/2}$ for $n$ even.
\end{enumerate}
(Compare Property \ref{pro:09.04} and Corollary \ref{cor:15.08}.) For the corresponding result with integer coefficients, see problem~\ref{prob:15-C}. 

\begin{proof}[Proof by induction on $n$]
For $n = 1$ the space ${\xtilde \grassmannian}_1({\mathbb R}^N)$ is clearly homeomorphic to the unit sphere $\sphere^{N - 1}$, and hence has the cohomology of a point in dimensions $\le N - 2$. Passing to the direct limit as $N \to \infty$, it follows that ${\xtilde \grassmannian}_1$ has the cohomology of a point in all dimensions.

Suppose inductively that the Theorem has already been verified for ${\xtilde \grassmannian}_{n - 1}$. Just as in the complex case (Theorem \ref{thm:14.05}), there is an exact sequence\index{Gysin sequence} \[\cdots \varrightarrow{} \homology^i({\xtilde \grassmannian}_n) \varrightarrow{\smile \eulerclass} \homology^{i + n}({\xtilde \grassmannian}_n) \varrightarrow{\lambda} \homology^{i + n}({\xtilde \grassmannian}_{n - 1}) \varrightarrow{} \homology^{i + 1}({\xtilde \grassmannian}_n) \varrightarrow{} \cdots\] where $\eulerclass$ stands for the Euler class $\eulerclass({\xtilde \gamma}^n)$, and where the ring homomorphism $\lambda = {f^\ast}^{-1} \pi_0^\ast$ maps the Pontrjagin classes of ${\xtilde \gamma}^n$ into those of ${\xtilde \gamma}^{n - 1}$. The coefficient ring $\Lambda$ is to be understood. 
\begin{enumerate}[label = Case \arabic*.]
    \item If $n$ is even, then the argument is completely analogous to that in Theorem \ref{thm:14.05}. This given exact sequence reduces to \[0 \varrightarrow{} \homology^i({\xtilde \grassmannian}_n) \varrightarrow{\smile \eulerclass} \homology^{i + n}({\xtilde \grassmannian}_n) \varrightarrow{\lambda} \homology^{i + n}({\xtilde \grassmannian}_{n - 1}) \varrightarrow{} 0,\] where the cohomology of ${\xtilde \grassmannian}_{n - 1}$ is a polynomial ring generated by $\pontrjaginclass_1, \ldots, \pontrjaginclass_{(n/2) - 1}$. It follows easily that $\homology^\ast({\xtilde \grassmannian}_n)$ is a polynomial ring on the required generators $\pontrjaginclass_1, \ldots, \pontrjaginclass_{(n/2) - 1}$, and $\eulerclass$. 
    
    \item Suppose that $n$ is odd, say $n = 2 m + 1$. Then the Euler class of ${\xtilde \gamma}^n$ with coefficients in $\Lambda$ is zero, so the exact sequence reduces to \[0 \varrightarrow{} \homology^j({\xtilde \grassmannian}_{2m + 1}) \varrightarrow{\lambda} \homology^j({\xtilde \grassmannian}_{2m}) \varrightarrow{} \homology^{j - 2m}({\xtilde \grassmannian}_{2m + 1}) \varrightarrow{} 0.\] Thus $\homology^\ast({\xtilde \grassmannian}_{2m + 1})$ can be considered as a sub--ring of $\homology^\ast({\xtilde \grassmannian}_{2m})$. 
    
    It will be convenient to introduce the abbreviation $A^\ast$ for the polynomial algebra $\Lambda[\pontrjaginclass_1, \ldots, \pontrjaginclass_m] \subset \homology^\ast({\xtilde \grassmannian}_{2m})$. Then clearly \[A^\ast \subset \lambda(\homology^\ast({\xtilde \grassmannian}_{2m + 1})),\] and we must prove that equality holds. It follows of course that the inequality
\begin{equation}
\label{eqn:15.1}
\rank A^j \le \rank \homology^j({\xtilde \grassmannian}_{2m + 1})
\end{equation}
is satisfied for each dimension $j$. (Here the \defemphi{rank} of a $\Lambda$--module means the maximal number of elements linearly independent over $\Lambda$. Compare \cite[p. 52]{eilenbergsteenrod1952}.)

Using the induction hypothesis we see easily that every element of $\homology^j({\xtilde \grassmannian}_{2m})$ can be written uniquely as a sum $a + \eulerclass a'$ with $a \in A^j$ and $a' \in A^{j - 2m}$. (Here $\eulerclass$ denotes the Euler class ${\xtilde \gamma}^{2m}$, with $\eulerclass^2 = \pontrjaginclass_m$.) This direct sum decomposition $\homology^j({\xtilde \grassmannian}_{2m}) \cong A^j \oplus A^{j - 2m}$ implies that
\begin{equation}
\label{eqn:15.2}
\rank \homology^j({\xtilde \grassmannian}_{2m}) = \rank A^j + \rank A^{j - 2m}.
\end{equation}
On the other hand, using the exact sequence above we see that 
\begin{equation}
\label{eqn:15.3}
\rank \homology^j({\xtilde \grassmannian}_{2m}) = \rank \homology^j({\xtilde \grassmannian}_{2m + 1}) + \rank \homology^{j - 2m}({\xtilde \grassmannian}_{2m + 1}).
\end{equation}
Combining \eqref{eqn:15.1}, \eqref{eqn:15.2} and \eqref{eqn:15.3}, it follows that \[\rank A^j = \rank \homology^j({\xtilde \grassmannian}_{2m + 1}).\] But this implies that $A^j$ is actually equal to the image $\lambda(\homology^j({\xtilde \grassmannian}_{2m + 1}))$. For otherwise $\lambda(\homology^j({\xtilde \grassmannian}_{2m + 1}))$ would contain a sum $a + \eulerclass({\xtilde \gamma}^{2m})a'$ with $a' \ne 0$. This new element could not satisfy any linear relation with the basis elements of $A^j$, so strict inequality would have to hold in \eqref{eqn:15.1}, yielding a contradiction. 
\end{enumerate}


\end{proof}

\begin{problem}
\label{prob:15-A}
Using Problem~\ref{prob:14.B}, prove that the mod $2$ reduction of the Pontrjagin class $\pontrjaginclass_i(\xi)$ is equal to the square of the Stiefel--Whitney class $\sw_{2i}(\xi)$.\index{Stiefel-Whitney class $\sw_i$}
\end{problem}

\begin{problem}
\label{prob:15-B}
Show that $\homology^\ast(\grassmannian_n({\mathbb R}^\infty); \Lambda)$ is a polynomial ring over $\Lambda$ generated by the Pontrjagin classes $\pontrjaginclass_1(\gamma^n), \ldots, \pontrjaginclass_{\lfloor n/2\rfloor}(\gamma^n)$. [More generally, for any $2$--fold covering space\index{covering space} ${\xtilde X} \longrightarrow X$ with covering transformation $t : {\xtilde X} \longrightarrow {\xtilde X}$, show that $\homology^\ast(X; \Lambda)$ can be identified with the fixed point set of the involution $t^\ast$ of $\homology^\ast({\xtilde X}; \Lambda)$.]
\end{problem}

\begin{problem}
\label{prob:15-C}\index{cohomology!\indexline of $\grassmannian_n$}
Compute the cohomology of the cochain complex\index{chain complex, mapping} \newline $\homology^\ast(\grassmannian_{2m + 1}({\mathbb R}^\infty); {\mathbb Z}/2)$ with respect to the differential operator $\Sq^1$\index{Steenrod squares}. [That is compute $\ker(\Sq^1)/\mathrm{Im}(\Sq^1)$. It is convenient to express this cohomology ring as the tensor product of a polynomial ring generated by $\sw_1$, and the polynomial rings generated by $\sw_{2i}$ and $\Sq^1(\sw_{2i})$ for $1 \le i \le m$.] Using the Bockstein exact sequence\index{Bockstein homomorphism} \[\cdots \varrightarrow{} \homology^j(-; \mathbb Z) \varrightarrow{2} \homology^j(-; \mathbb Z) \varrightarrow{\rho} \homology^j(-; {\mathbb Z}/2) \varrightarrow{\beta} \homology^{j + 1}(-; \mathbb Z) \varrightarrow{} \cdots,\] where $\rho \circ \beta = \Sq^1$ (compare \cite[p. 2]{eilenbergsteenrod1952}), prove that $\homology^\ast(\grassmannian_{2m + 1}({\mathbb R}^\infty); \mathbb Z)$ splits additively as the direct sum of the polynomial ring $\mathbb Z[\pontrjaginclass_1, \ldots, \pontrjaginclass_m]$ and the image of $\beta$. Prove the analogous statements for $\grassmannian_{2m}({\mathbb R}^\infty)$ and ${\xtilde \grassmannian}_n({\mathbb R}^\infty)$. 
\end{problem}

\begin{problem}
\label{prob:15-D}\index{Chern class $\chernclass_i$}
Using the preceding, prove that the odd Chern classes of $\xi \otimes \mathbb C$ are given by \[\chernclass_{2i}(\xi \otimes \mathbb C) = \beta(\sw_{2i}(\xi)\sw_{2i + 1}(\xi)).\] Similarly, for an oriented $(2k + 1)$--plane bundle $\xi$, prove that $\eulerclass(\xi) = \beta \sw_{2k}(\xi)$. 
\end{problem}
\end{document}