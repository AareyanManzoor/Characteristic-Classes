\documentclass[../main]{subfiles}
\begin{document}
\section{Chern Numbers}
Let $K^n$ be a compact complex manifold of complex dimension $n$. Then for each partition $I = i_1, \ldots, i_r$ of $n$, the \defemph{$I$--th Chern number} \[\chernclass_i[K^n] = \chernclass_{i_1} \cdots \chernclass_{i_r}[K^n]\] is defined to be the integer \[\ip {\chernclass_{i_1} (\tau^n) \ldots \chernclass_{i_r}(\tau^n)} {\mu_{2n}}.\] Here $\tau^n$ denotes the tangent bundle of $K^{2n}$, and $\mu_{2n}$ denotes the fundamental homology class\index{fundamental class!\indexline homology} determined by the preferred orientation. We adopt the convention that $\chernclass_I[K^n]$ is zero if $I$ is a partition of some integer other than $n$. 

As an example, for the complex projective space ${\mathbb P}^n(\mathbb C)$,\index{projective space!\indexline real $\projective^n$} since $\chernclass_i(\tau^n) = \binom {n + 1} i a^i$ and $\ip {a^n} {\mu_{2n}} = +1$ by Theorem \ref{thm:14.10}, we have the formula\index{binomial coefficients} \[\chernclass_{i_1} \cdots \chernclass_{i_r} [{\mathbb P}^n(\mathbb C)] = \binom {n + 1} {i_1} \cdots \binom {n + 1} {i_r}\] for any partition $i_1, \ldots, i_r$ of $n$. 

A complex $1$--dimensional manifold $K^1$ has just one Chern number, namely the Euler characteristic\index{Euler characteristic} $\chernclass_1[K^1]$. For a complex $2$--manifold there are two Chern numbers, namely $\chernclass_1 \chernclass_1[K^2]$ and the Euler characteristic $\chernclass_2[K^2]$. In general, a complex $n$--manifold has $p(n)$ Chern numbers, where $p(n)$ is the number of distinct partitions of $n$. (Compare p. \pageref{def:06.06}.) We will see in \ref{thm:16.07} that these $p(n)$ Chern numbers are linearly independent; that is there is no linear relation between them which is satisfied for \defemph{all} complex $n$--manifolds. 

There is another way of thinking about Chern classes which is important for many purposes. Note that the cohomology group $\homology^{2n}(\grassmannian_n({\mathbb C}^\infty); \mathbb Z)$ is free abelian of rank $p(n)$. The products $\chernclass_{i_1}(\gamma^n) \ldots \chernclass_{i_r}(\gamma^n)$, where $i_1, \ldots, i_r$ ranges over all partitions of $n$, form a basis for this group. For any complex manifold $K^n$ the tangent bundle $\tau^n$ is ``classified'' by a map \[f : K^n \longrightarrow \grassmannian_n({\mathbb C}^\infty)\] with $f^\ast(\gamma^n) \cong \tau^n$. Using this classifying map $f$, the fundamental homology class $\mu_{2n}$ of $K^n$ gives rise to a homology class $f_\ast(\mu_{2n})$ in the free abelian group $\homology_{2n}(\grassmannian_n({\mathbb C}^\infty); \mathbb Z)$ of rank $p(n)$. To identify this homology class $f_\ast(\mu_{2n})$, we only need to compute the $p(n)$ Kronecker indices \[\ip {\chernclass_{i_1}(\gamma^n) \ldots \chernclass_{i_r}(\gamma^n)} {f_\ast(\mu_{2n})},\] since the products $\chernclass_{i_1}(\gamma^n) \ldots \chernclass_{i_r}(\gamma^n)$ range over a basis for the corresponding cohomology group. But each such Kronecker index is equal to the Chern number \[\ip {f^\ast(\chernclass_{i_1}(\gamma^n) \ldots \chernclass_{i_r}(\gamma^n))} {\mu_{2n}} = \chernclass_{i_1} \cdots \chernclass_{i_r}[K^n].\] We see from this approach that it is not necessary to use the basis $\{\chernclass_{i_1}(\gamma^n) \ldots \chernclass_{i_r}(\gamma^n)\}$ for $\homology^{2n}(\grassmannian_n({\mathbb C}^\infty); \mathbb Z)$. Any other basis would serve equally well. Later we will make use of a quite different basis for this group. 
\end{document}