\documentclass[../main]{subfiles}
\begin{document}
\section{Partitions}
%check "denote by the empty symbol"
Recall from definition \ref{def:06.06} in $\S6$, that a \defemphi{partition} of a non-negative integer $k$ is an unordered sequence $I = i_1, \ldots, i_r$ of positive integers with sum $k$. If $I = i_1, \ldots, i_r$ is a partition of $k$ and $J = j_1, \ldots, j_s$ is a partition of $\ell$, then the juxtaposition \[IJ = i_1, \ldots, i_r, j_1, \ldots, j_s\] is a partition of $k + \ell$. This composition operation is associative, commutative, and has as identity element the vacuous partition of zero which we denote by the empty symbol \phantom{empty}. (In more technical language, the set of all partitions of all non-negative integers can be regarded as a free commutative monoid on the generators $1, 2, 3, \ldots$ .)

A partial ordering among partitions is defined as follows. A \defemphi{refinement} of a partition $i_1, \ldots, i_r$ will mean any partition which can be written as a juxtaposition $I_1, \ldots, I_r$ where each $I_j$ is a partition of $i_j$. If $j_1, \ldots, j_s$ is a refinement of $i_1, \ldots, i_r$ then it follows of course that $s \ge r$. 
\end{document}