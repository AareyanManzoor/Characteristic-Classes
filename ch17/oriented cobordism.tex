\documentclass[../main]{subfiles}
\begin{document}
\section{Oriented Cobordism}
If $M$ is a smooth oriented manifold, then the notation $-M$ will be used for the same manifodl with opposite orientation. The symbol $+$ will be used for the disjoint union (also called topological sum) of smooth manifolds.
\begin{definition}
Two smooth compact oriented $n$-dimensional manifolds $M$ and $M'$ are said to be \defemph{oriented cobordant}, or to belong to the same \defemph{oriented cobordism class}, if there exists a smooth, compact, oriented manifold-with-boundary $X$ so that $\partial X$ with its induced orientation is diffeomorphic to $M + (-M')$ under an orientation preserving diffeomorphism.
\end{definition}
\begin{lemma}
\label{lem:17.02}
This relation of oriented cobordism is reflexive, symmetric, and transitive.
\end{lemma}
Indeed, the disjoint union $M+(-M)$ is certainly diffeomorphic to the boundary of $[0,1] \times M$ under an orientation preserving diffeomorphism. Furthermore, if $M+(-M') \cong \partial X$, then clearly $M' + (-M) \cong \partial(-X)$. Finally, if $M+(-M') \cong \partial X$ and $M' + (-M'') \cong \partial Y$, then using \ref{thm:17.1} the smoothness structures and the orientations of $X$ and $Y$ can be pieced together along with common boundary $M'$ so as to yield a new smooth oriented manifold-with-boundary bounded by $M+(-M'')$. Details will be left to the reader.\ensuremath{\blacksquare}

Now the set $\orientedCobordism_n$ consisting of all oriented cobordism classes of $n$-dimensional manifolds clearly forms an abelian group, using the disjoint union $+$ as composition operation. The zero element of this group is the cobordism class of the vacuous manifold.

Furthermore the cartesian product operation $M_1^m, M_2^n \mapsto M_1^m \times M_2^n$ gives rise to an associative, bilinear product operation \[\orientedCobordism_m \times \orientedCobordism_n \rightarrow \orientedCobordism_{m+n}.\] \defemph{Thus the sequence} \[\orientedCobordism_\ast = (\orientedCobordism_0, \orientedCobordism_1, \orientedCobordism_2,\cdots)\] \defemph{of oriented cobordism groups has the structure of a graded ring.} This ring possesses a $2$-sided identity element $1 \in \orientedCobordism_0$. Furthermore, it is easily verified that $M_1^m \times M_2^n$ is isomorphic as oriented manifold to $(-1)^{mn}M_2^n \times M_1^m$. Thus this \defemph{oriented cobordism ring} is commutative in the graded sense.

Pontrjagin numbers\index{Pontrjagin number} provide a basic tool for studying these cobordism groups. As already pointed out in \S\ref{ch:16}, we have the following statement.
\begin{lemma}[Pontrjagin]\index{Pontrjagin, L.}
\label{lem:17.03}
If $M^{4k}$ is the boundary of a smooth, compact, oriented $(4k+1)$-dimensional manifold with-boundary, then every Pontrjagin number $\pontrjaginclass_{i_1} \cdots \pontrjaginclass_{i_r}[M^{4k}]$ is zero.
\end{lemma}
Since the identity $\pontrjaginclass_I[M_1 + M_2] = \pontrjaginclass_I[M_1] + \pontrjaginclass_I[M_2]$ is clearly satisfied, this proves the following.
\begin{corollary}
\label{cor:17.04}
For any partition\index{partition} $I = i_1,\cdots, i_r$ of $k$, the correspondence $M^{4k} \mapsto \pontrjaginclass_I[M^{4k}]$ gives rise to a homomorphism from the cobordism group $\Sigma_{4k}$ to $\mathbb{Z}$.
\end{corollary}

Now by \ref{thm:16.08} we obtain the following.

\begin{corollary}
\label{cor:17.05}
The products $\projective^{2i_1}(\bC) \times \cdots \times \projective^{2i_r}(\bC)$\index{projective space!\indexline complex $\projective^n(\bC)$}, where $i_1,\cdots,i_r$ ranges over all partitions of $k$, represent linearly independent elements of the cobordism group $\orientedCobordism_{4k}$. Hence $\orientedCobordism_{4k}$ has rank greater than or equal to $p(k)$, the number of partitions of $k$.
\end{corollary}
Following Thom, we will prove in \S\ref{ch:18} that the rank is precisely $p(k)$,

To conclude this section, we list without proof the actual structures of the first few oriented cobordism groups. (Compare \cite[p. 309]{wall}.)
\allowdisplaybreaks

\begin{tabular}{ll}
$\orientedCobordism_0 \cong \mathbb{Z}$.     & In fact a compact oriented $0$-manifold is just a finite set of signed \\
& points, nd the sum of the signs is a complete cobordism invariant.\\
$\orientedCobordism_1 = 0$,     & since every compact $1$-manifold clearly bounds. \\
$\orientedCobordism_2 = 0$,     & since a compact \defemph{oriented} $2$-manifold bounds. \\
$\orientedCobordism_3 = 0$.     & In contrast to the lower dimensional cases, this assertion, first\\
&announced by \cite{rokhlin}, is non-trivial. To our knowledge it has\\
&never been proven directly. \\ 
$\orientedCobordism_4 \cong \mathbb{Z}$,    & generated by the complex projective plane $\projective^2(\bC)$. \\ 
$\orientedCobordism_5 \cong \mathbb{Z}/2$,      & generated by the manifold $Y^5$ of Problem \ref{prob:16.F}. \\
$\orientedCobordism_6 = 0.$ & \\
$\orientedCobordism_7 = 0.$ & \\
$\orientedCobordism_8 \cong \mathbb{Z} \oplus \mathbb{Z},$ & generated by $\projective^4(\bC)$ and $\projective^2(\bC)\times\projective^2(\bC)$ \\ 
$\orientedCobordism_9 \cong (\mathbb{Z}/2) \oplus (\mathbb{Z}/2),$ & generated by $Y^9$ and the product $Y^5 \times \projective^2(\bC)$. \\ 
$\orientedCobordism_{10} \cong \mathbb{Z}/2$, & generated by $Y^5 \times Y^5$. \\
$\orientedCobordism_{11} \cong \mathbb{Z}/2$, & generated by $Y^{11}$.
\end{tabular}

As manifold $Y^5$ (respectively $Y^9, Y^{11}$) we may take the non-singular hypersurface of degree $(1,1)$ in the product $\projective^2 \times \projective^4$ (respectively $\projective^2 \times \projective^8$ or $\projective^4 \times \projective^8$) of real projective spaces. Using products of the generators listed above, it is easy to show that all of the higher cobordism groups are non-zero.

\end{document}