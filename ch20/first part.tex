\documentclass[../main]{subfiles}
\begin{document}
For any triangulated manifold $M^n$, \cite{thom1968}\index{Thom, R.} has defined classes $\ell_i \in \homology^{4i}(M^n; \mathbb Q)$ which are combinatorial (i.e., piecewise linear) invariants. (See also \cite{rokhlin1957}.) In the case of a smooth manifold, suitably triangulated, these coincide with the Hirzebruch classes\index{Hirzebruch, F.} $L_i(\pontrjaginclass_1, \ldots, \pontrjaginclass_i)$ of the tangent bundle $\tau^n$. \index{L-genus@$L$-genus}

Now recall (Problem \ref{prob:19.C}) that the coefficient of $\pontrjaginclass_i$ in the polynomial $L_i(\pontrjaginclass_1, \ldots, \pontrjaginclass_i)$ is non--zero. Hence it follows by induction that the equations $\ell_i = L_i(\pontrjaginclass_1, \ldots, \pontrjaginclass_i)$ can be uniquely solved for the Pontrjagin classes $\pontrjaginclass_i$ as polynomial functions of $\ell_1, \ldots, \ell_i$. For example
\begin{align*}
\pontrjaginclass_1 & = 3 \ell_1, \\ \pontrjaginclass_2 & = \dfrac{45 \ell_2 + 9 \ell_1^2}{7},
\end{align*}
and so on. \defemph{Thus it follows that the rational Pontrjagin classes \newline $\pontrjaginclass_i(\tau^n) \in \homology^{4i}(M^n; \mathbb Q)$ are piecewise linear invariants.} This section contains an exposition of these results. 

In 1965 \cite{novikov1966} proved the much sharper statement that rational Pontrjagin classes are \defemph{topological} invariants. (Compare the \hyperref[ch:21]{Epilogue}) We will not try to discuss this sharper theorem.
\end{document}