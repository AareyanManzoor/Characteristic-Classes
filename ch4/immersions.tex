\documentclass[../main]{subfiles}
\begin{document}
\section{Immersions}\label{sec:4.3}

As a final application of Theorem~\ref{thm:04.05}, let us ask which projective spaces can be immersed in the Euclidean space of a given dimension.

If a manifold $M$ of dimension $n$ can be immersed\index{immersion} in the Euclidean space $\bR^{n+k}$ then the Whitney duality theorem\index{Whitney duality theorem}
\[
\sw_{i}(\nu)=\xoverline{\sw}_{i}(M)
\]
implies that the dual Stiefel-Whitney classes\index{Stiefel-Whitney class $\sw_i$!\indexline dual} $\xoverline{\sw}_{i}(M)$ are zero for $i>k$.

As a typical example, consider the real projective space $\projective^{9}$\index{projective space!\indexline real $\projective^n$}. Since
\[
\sw(\projective^{9}):=(1+a)^{10}=1+a^{2}+a^{8}
\]
we have
\[
\xoverline{\sw}(\projective^{9})=1+\mathrm{a}^{2}+\mathrm{a}^{4}+\mathrm{a}^{6}
\]
Thus if $\projective^{9}$ can be immersed in $\bR^{9+k}$, then $k$ must be at least $6$.

The most striking results for $\projective^{n}$ are obtained when $n$ is a power of $2$. If $n=2^{r}$ then
\[
\sw(\projective^{n})=(1+a)^{n+1}:=1+a+a^{n}
\]
hence
\[
\xoverline{\sw}\left(\projective^{n}\right)=1+a+a^{2}+\dots+a^{n-1}.
\]
Thus:
\begin{theorem}\label{thm:04.08}
	If $\projective^{2^{r}}$ can be immersed in $\bR^{2^{r}+k}$, then $k$ must be at least $2^{r}-1$.
\end{theorem}

On the other hand Whitney has proved that every smooth compact manifold of dimension $n>1$ can actually be immersed in $\bR^{2 n-1}$. (See \cite{whitney1944}\index{Whitney, H.}.) Thus \ref{thm:04.08} provides a best possible estimate.

Note that estimates for other projective spaces follow from \ref{thm:04.08}. For example since $\projective^{8}$ cannot be immersed in $\bR^{14}$, it follows a fortiori that $\projective^{9}$ cannot be immersed in $\bR^{14}$. This duplicates the earlier estimate concerning $\projective^{9}$. See \cite{james1971}.

An extensive and beautiful theory concerning immersions of manifolds has been developed by S. Smale and M. Hirsch. For further information the reader should consult \cite{hirsch1959} and \cite{smale1959}.

\end{document}