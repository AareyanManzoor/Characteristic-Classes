\documentclass[../main]{subfiles}

\begin{document}

In classical differential geometry one encounters the ``spherical image''
of a curve $M^1\subset \mathbb R^{k+1}$. This is the image of $M^1$ under the mapping
\[t : M^1 \varrightarrow{} S^k\]
which carries each point of $M^1$ to its unit tangent vector. Similarly Gauss\index{Gauss map} defined the spherical image of a hypersurface $M^k \subset \mathbb R^{k+1}$ as the image of $M^k$ under the mapping
\[n : M^k \varrightarrow{} S^k\]
which carries each point of $M$ to its unit normal vector. (Compare figure \ref{fig:figure6},\ref{fig:figure7}.) In order to specify the sign of the tangent or normal vector it is
necessary to assume that $M^1$ or $M^k$ has a preferred orientation\index{oriented manifold}. (Compare
\S\ref{ch:9}.) However without this orientation one can still define a corresponding
map from the manifold to the real projective space $\projective^k$.\index{projective space!\indexline real $\projective^n$}

\begin{figure}[ht]
    \centering
    \incfig{fig7}
    \caption{}
    \label{fig:figure7}
\end{figure}
\begin{figure}[ht]
    \centering
    \incfig{fig8}
    \caption{}
    \label{fig:figure8}
\end{figure}

More generally let $M$ be a smooth manifold of dimension $n$ in the coordinate space $\mathbb{R}^{n+k}$. Then to each point $x$ of $M$ one can assign the tangent space \newline $\tangentspace{M}{x} \subset \mathbb{R}^{n+k}$. We will think of $\tangentspace{M}{x}$ as determining a point in a new topological space $\grassmannian_n (\mathbb{R}^{n+k})$.
\begin{definition}\label{def:5-1}
The \defemph{Grassmann manifold} $\grassmannian_n (\mathbb{R}^{n+k})$ is the set of all $n$-dimensional planes through the origin of the coordinate space $\mathbb{R}^{n+k}$. This is to be topologized as a quotient space, as follows.
An $n$-frame\index{frame}\index{n-frame@$n$-frame} in $\mathbb{R}^{n+k}$ is an $n$-tuple of linearly independent vectors of $\mathbb{R}^{n+k}$. The collection of all $n$-frames in $\mathbb{R}^{n+k}$ forms an open subset of the n-fold Cartesian product $\mathbb{R}^{n+k} \times \dots \times \mathbb{R}^{n+k}$, called the \defemphi{Stiefel manifold} $\StiefelManifold_n(\mathbb{R}^{n+k}) $. (Compare \cite[$\S$7.7]{steenrod1951}.) There is a canonical function
\[q:\StiefelManifold_n(\mathbb{R}^{n+k})\varrightarrow{}\grassmannian_n (\mathbb{R}^{n+k}) \]
which maps each $n$-frame to the $n$-plane\index{n-plane@$n$-plane} which it spans. Now give $\grassmannian_n (\mathbb{R}^{n+k})$ the quotient topology: a subset $U \subset \grassmannian_n (\mathbb{R}^{n+k})$ is open if and only if its inverse image $q^{-1}(U) \subset \StiefelManifold_n(\mathbb{R}^{n+k})$ is open.

Alternatively let $\StiefelManifold_n^0(\mathbb{R}^{n+k})$ denote the subset of $\StiefelManifold_n(\mathbb{R}^{n+k})$ consisting of all orthonormal $n$-frames, Then $\grassmannian_n (\mathbb{R}^{n+k})$ can also be considered as an identification space of $\StiefelManifold_n^0(\mathbb{R}^{n+k})$. One sees from the following
commutative diagram that both constructions yield the same topology for $\grassmannian_n (\mathbb{R}^{n+k})$.
\index{Gram-Schmidt process}
\[\begin{tikzcd}
	\StiefelManifold_n^0(\mathbb{R}^{n+k}) \arrow[r, hook] \arrow[rd, "q_0"'] & \StiefelManifold_n(\mathbb{R}^{n+k}) \arrow[d, "q"'] \arrow[rrrr, "\text{Gram-Schmidt Process}"] &  &  & & \StiefelManifold_n^0(\mathbb{R}^{n+k}) \arrow[lllld, "q_0"] \\
	& \grassmannian_n (\mathbb{R}^{n+k})                                                         &  &  &   &                                     
\end{tikzcd}\]
Here $q_0$ denotes the restriction of $q$ to $\StiefelManifold_n^0(\mathbb{R}^{n+k})$.	
\end{definition}
\begin{lemma}\label{lem:05.01}
	The Grassmann manifold $\grassmannian_n (\mathbb{R}^{n+k})$ is a compact topological manifold\index{topological manifold}\footnote{ A \defemph{topological manifold} of dimension $d$ is a Hausdorff space in which every
		point has a neighborhood homeomorphic to $\mathbb{R}^d$.} of dimension $nk$. The correspondence $X \rightarrow X^{\perp}$, which assigns to each $n$-plane its orthogonal $k$-plane, defines a homeomorphism between $\grassmannian_n (\mathbb{R}^{n+k})$ and $\grassmannian_{k}(\mathbb{R}^{n+k})$.
\end{lemma}
\begin{remark*}
	For the special case $k=1$ note that $\grassmannian_{1}(\mathbb{R}^{n+1})$ is equal to the real projective space $\projective^n$. It follows that the manifold $\grassmannian_{n}(\mathbb{R}^{n+1})$ of $n$-planes in $(n+1)$-space is canonically homeomorphic to $\projective^n$.
\end{remark*}

\begin{proof}[Proof of \ref{lem:05.01}]
	In order to show that $\grassmannian_n (\mathbb{R}^{n+k})$ is a Hausdorff space it is sufficient to show that any two points can be separated by a continuous real valued function. For fixed $w \in \mathbb{R}^{n+k}$, let $\rho_{w}(X)$ denote the square of the Euclidean distance from $w$ to $X$. If $x_{1}, \dots, x_{n}$ is an orthonormal basis for $X$, then the identity
	$$
	\rho_{w}(X)=w \cdot w-(w \cdot x_{1})^{2}-\dots-(w \cdot x_{n})^{2}
	$$
	shows that the composition
	\[
	\StiefelManifold_n^0(\mathbb{R}^{n+k}) \varrightarrow{q_0}\grassmannian_n (\mathbb{R}^{n+k}) \varrightarrow{\rho_w}  \mathbb{R}
	\]
is continuous; hence that $\rho_{w}$ is continuous. Now if $X, Y$ are distinct $n$-planes, and $w$ belongs to $X$ but not $Y$, then $\rho_{w}(X) \neq \rho_{w}(Y)$. This proves that $\grassmannian_n (\mathbb{R}^{n+k})$ is a Hausdorff space.

The set $\StiefelManifold_n^0(\mathbb{R}^{n+k})$ of orthonormal $n$-frames is a closed, bounded subset of $\mathbb{R}^{n+k} \times \dots \times \mathbb{R}^{n+k}$, and therefore is compact. It follows that
\[\grassmannian_n (\mathbb{R}^{n+k})=q_{0}(\StiefelManifold_n^0(\mathbb{R}^{n+k}))\]
is also compact.\\
\begin{proof}\textit{that every point $X_{0}$ of $\grassmannian_n (\mathbb{R}^{n+k})$ has a neighborhood $U$ which is homeomorphic to $\mathbb{R}^{nk}$.}
 It will be convenient to regard $\mathbb{R}^{n+k}$ as the direct sum $X_{0} \oplus X_{0}{ }^\perp$. Let $U$ be the open subset of $\grassmannian_n (\mathbb{R}^{n+k})$ consisting of all $n$-planes $Y$ such that the orthogonal projection
\[p:X_0\oplus X_{0}^{\perp}\varrightarrow{}X_0\]
maps $Y$ onto $X_{0}$ (i,e., all $Y$ such that $Y \cap X_{0}^{\perp}=0$ ). Then each $Y \in U$ can be considered as the graph of a linear transformation
\[T(Y):X_{0}\varrightarrow{}X_{0}^{\perp}.\]
This defines a one-to-one correspondence\index{Hom}
\[T:U \varrightarrow{}\Hom(X_0,X_{0}^{\perp})\cong\mathbb{R}^{nk}.\]
We will see that $T$ is a homeomorphism.

Let $x_{1}, \dots, x_{n}$ be a fixed orthonormal basis for $X_{0}$. Note that each $n$-plane $Y \in U$ has a unique basis $y_{1}, \dots, y_{n}$ such that
\[p(y_{1})=x_{1}, \dots, p(y_{n})=x_{n}.\]
It is easily verified that the $n$-frame $(y_{1}, \dots, y_{n})$ depends continuously on $Y$.

Now note the identity
\[y_{i}=x_{i}+T(Y) x_{i}.\]
Since $y_i$ depends continuously on $Y$, it follows that the image
$T(Y)x_i\in X_0^\perp$ depends continuously on $Y$. Therefore the linear 
transformation $T(Y)$ depends continuously on $Y$.

On the other hand this identity shows that the $n$-frame $(y_1,\dots, y_n) $
depends continuously on $T(Y)$, and hence that $Y$ depends continuously
on $T(Y)$. Thus the function $T\inv$ is also continuous. This completes
the proof that $\grassmannian_n (\mathbb{R}^{n+k})$ is a manifold.\\
\end{proof}
\begin{proof}[Proof that $Y^\perp$ depends continuously on $Y$.]
Let $(\xoverline{x}_1,\dots,\xoverline{x}_k)$ be a fixed basis for $X_{0}{ }^{\perp}$. Define a function
\[f:q\inv U \varrightarrow{}\StiefelManifold_k(\mathbb{R}^{n+k})\]
as follows. For each $(y_{1}, \dots, y_{n}) \in q^{-1} U$, apply the Gram-Schmidt process\index{Gram-Schmidt process} to the vectors $(y_{1}, \dots, y_{n}, \xoverline{x}_{1}, \dots, \xoverline{x}_{k})$; thus obtaining an orthonormal $(n+k)$-frame $(y_{1}^{\prime}, \dots, y_{n+k}^{\prime})$ with $y_{n+1}^{\prime}, \dots, y_{n+k}^{\prime} \in Y^{\perp}$. Setting $f(y_{1}, \dots, y_{n})=(y_{n+1}^{\prime}, \dots, y_{n+k}^{\prime})$, it follows that the diagram
\[\begin{tikzcd}
	q\inv U \arrow[r, "f"] \arrow[d, "q"'] & \StiefelManifold_k(\mathbb{R}^{n+k}) \arrow[d, "q"'] \\
	U \arrow[r, "\perp"]                   & \grassmannian_{k}(n+k)                    
\end{tikzcd}\]
is commutative. Now $f$ is continuous, so $q \circ f$ is continuous, therefore the correspondence $Y \mapsto Y^{\perp}$ must also be continuous. This completes the proof of \ref{lem:05.01}.
\end{proof}
\end{proof}
A canonical vector bundle\index{cannonical bundle!\indexline $n$-plane $\tautological^n$} $\gamma^n(\mathbb{R}^{n+k})$ over $\grassmannian_n (\mathbb{R}^{n+k})$ is constructed
as follows. Let
\[\total=\total(\gamma^n(\mathbb{R}^{n+k}))\]
be the set of all pairs\footnote{Here, and elsewhere, the expression ``$n$-plane'' means linear subspace of
	dimension $n$. Thus we only consider $n$-planes through the origin.}
\[
 (\text {$n$-plane in } \mathbb{R}^{n+k}, \text { vector in that $n$-plane}). 
\]
This is to be topologized as a subset of $\grassmannian_n (\mathbb{R}^{n+k})\times \mathbb{R}^{n+k}$. The projection map $\pi:\total\varrightarrow{} \grassmannian_n (\mathbb{R}^{n+k})$ is defined by $\pi(X, x)=X$, and the vector space structure in the fiber over $X$ is defined by $t_{1}(X, x_{1})+t_{2}(X, x_{2})= (X, t_{1} x_{1}+t_{2} x_{2}) .$ (Note that $\gamma^{1}(\mathbb{R}^{n+1})$ is the same as the line bundle $\gamma_{n}^{1}$ described in \S\ref{ch:2}.)

\begin{lemma}\label{lem:05.02}
	The vector bundle $\gamma^{n}(\mathbb{R}^{n+k})$ constructed in this way satisfies the local triviality condition.
\end{lemma}
\begin{proof}
	Let $U$ be the neighborhood of $X_{0}$ constructed as in Lemma~\ref{lem:05.01}. Define the coordinate homeomorphism
	\[h:U\times X_0\varrightarrow{}\pi\inv (U)
	\]
	as follows. Let $h(Y, x)=(Y, y)$ where $y$ denotes the unique vector in $Y$ which is carried into $x$ by the orthogonal projection
	\[p:\mathbb{R}^{n+k} \varrightarrow{}X_{0}. 
	\]
	The identities
	\[
	h(Y, x)=(Y, x+T(Y) x)
	\]
	and
	\[
	h^{-1}(Y, y)=(Y, py)
	\]
show that $h$ and $h^{-1}$ are continuous. This completes the proof of \ref{lem:05.02}.
\end{proof}

Given a smooth $n$-manifold $M \subset \mathbb{R}^{n+k}$ the \defemph{generalized Gauss map}\index{Gauss map!\indexline generalized}

\[\map{\xoverline{g}}{M}{\grassmannian_n (\mathbb{R}^{n+k})},
\]
can be defined as the function which carries each $x \in M$ to its tangent space $\tangentspace{M}{x} \in \grassmannian_n (\mathbb{R}^{n+k})$. This is covered by a bundle map
\[\map{g}{\total(\tangentbundle{M})}{\total(\gamma^{n}(\mathbb{R}^{n+k}))}.
\]
where $g(x, v)=(\tangentspace{M}{x}, v) $. We will use the abbreviated notation
\[\map{g}{\tangentbundle{M}}{\gamma^{n}(\mathbb{R}^{n+k})}.
\]
It is clear that both $g$ and $\xoverline{g}$ are continuous.

Not only tangent bundles, but most other $\mathbb{R}^{n}$-bundles can be mapped into the bundle $\gamma^{n}(\mathbb{R}^{n+k})$ providing that $k$ is sufficiently large. For this reason $\gamma^{n}(\mathbb{R}^{n+k})$ is called a \defemphi{universal bundle}. (Compare Theorems~\ref{thm:05.06} and \ref{thm:05.07}, as well as \cite[$\S$ 19]{steenrodwhitehead1951}.)

\begin{lemma}\label{lem:05.03}
	For any $n$-plane\index{n-plane@$n$-plane} bundle $\xi$ over a compact base space $\B $ there exists a bundle map\index{bundle map} $\xi \rightarrow \gamma^{n}(\mathbb{R}^{n+k})$ provided that $k$ is sufficiently large.
\end{lemma}

In order to construct a bundle map $\map{f}{\xi}{\gamma^{n}(\mathbb{R}^{m})}$ it is sufficient to construct a map
\[\map{\hat{f}}{\total(\xi)}{\mathbb{R}^{m}}
\]
which is linear and injective (i.e., has kernel zero) on each fiber of $\xi$. The required function $f$ can then be defined by
\[
f(e)=(\hat{f} (\text{fiber through } e), \hat{f}(e)).
\]
The continuity of $f$ is not difficult to verify, making use of the fact that $\xi$ is locally trivial.

\begin{proof}[Proof of \ref{lem:05.03}]
Choose open sets $U_{1}, \dots, U_{r}$ covering $\B $ so that each $\xi \mid _{U_{i}}$ is trivial. Since $\B $ is normal, there exist open sets $V_{1}, \dots, V_{r}$ covering $\B $ with $\xoverline{V}_{i} \subset U_{i}$. (Compare \cite[p. 171]{kelley1955}.) Here $\xoverline{V}_{i}$ denotes the closure of $V_{i} .$ Similarly construct $W_{1}, \dots, W_{r}$ with $\xoverline{W}_{i} \subset V_{i} $. By Urysohn's lemma (Compare \cite[\S33]{munkres2000topology}) we have continuous functions
\[\map{\lambda_{i}}{\B }{\mathbb{R}}
\]
which takes the value 1 on $\xoverline{W}_{i}$ and the value 0 outside of $V_{i}$.

Since $\xi|_{ U_{i}}$ is trivial there exists a map
\[\map{h_{i}}{\pi^{-1} (U_{i})}{\mathbb{R}^{n}}
\]
which maps each fiber of $\xi| _{U_{i}}$ linearly onto $\mathbb{R}^{n}$. Define $\map{h_{i}^{\prime}}{\total(\xi)}{\mathbb{R}^{n}}$ by

\[
h_{i}^{\prime}(e)=\begin{cases}
	0 & \text { for } \pi(e) \notin V_{i} \\
	\lambda_{i}(\pi(e)) h_{i}(e) & \text { for } \pi(e) \in U_{i}
\end{cases}
\]
Evidently $h_{i}^{\prime}$ is continuous, and is linear on each fiber. Now define

\[\map{\hat{f}}{\total(\xi)}{\mathbb{R}^{n} \oplus \dots \oplus \mathbb{R}^{n} \cong \mathbb{R}^{rn}}
\]
by $\hat{f}(e)=(h_{1}^{\prime}(e), h_{2}^{\prime}(e), \dots, h_{r}^{\prime}(e))$. Then $\hat{f}$ is also continuous and maps each fiber injectively. This completes the proof of \ref{lem:05.03}.	
\end{proof}

\end{document} 