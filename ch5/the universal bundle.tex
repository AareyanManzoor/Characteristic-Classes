\documentclass[../main]{subfiles}

\begin{document}
\section{The Universal Bundle \texorpdfstring{$\gamma^n$}{gamma n}}\label{5.2}

A canonical bundle $\gamma^n$ over $\grassmannian_n$ is constructed, just as in the finite
dimensional case, as follows. Let 
\[
\total(\gamma^n)\subset \grassmannian_n\times {\mathbb R}^\infty
\]
be the set of all pairs
\[
(n\text{-plane in }{\mathbb R}^\infty,\text{ vector in that $n$-plane}),
\]
topologized as a subset of the Cartesian product. Define $\pi: \total(\gamma^n) \to \grassmannian_n$
by $\pi(X, x) = X$, and define the vector space structures in the fibers as
before.

\begin{lemma}\label{lem:05.04}
 This vector bundle $\gamma^n$ satisfies the local triviality
condition.
\end{lemma}
The proof will be essentially the same as that of Lemma~\ref{lem:05.02}. However the
following technical lemma will be needed. (Compare \cite[\S18.5]{whitehead1961}.)
\begin{lemma}\label{lem:05.05}\index{Cartesian product} Let $A_1 \subset A_2 \subset \cdots$ and $B_1 \subset B_2 \subset \cdots$ be sequences
of locally compact spaces with direct limits $A$ and $B$ respectively. Then the Cartesian product topology on $A \times B$ coincides
with the direct limit topology which is associated with the sequence $A_1 \times B_1 \subset A_2 \times B_2 \subset\cdots$.
\end{lemma}
\begin{proof} Let $W$ be open in the direct limit topology, and let $(a, b)$ be
any point of $W$. Suppose that $(a, b) \in A_i \times B_i$. Choose a compact neighborhood $K_i$ of $a$ in $A_i$ and a compact neighborhood $L_i$ of $b$ in $B_i$
so that $K_i \times L_i \subset W$. It is now possible (with some effort) to choose compact neighborhoods $K_{i+1}$ of $K_i$ in $A_{i+1}$ and $L_{i+1}$ of $L_i$ in $B_{i+1}$ so
that $K_{i+1} \times L_{i+1}\subset  W$. Continue by induction, constructing neighborhoods
$K_i \subset K_{i+1} \subset K_{i+2} \subset\cdots$ with union $U$ and $L_i \subset L_{i+1} \subset \cdots$ with union $V$.
Then $U$ and $V$ are open sets, and
\[
(a, b) \in U \times V \subset W.
\]
Thus $W$ is open in the product topology, which completes the proof of \ref{lem:05.05}.
\end{proof}
\begin{proof}[Proof of Lemma \ref{lem:05.04}] Let $X_0 \subset \mathbb{R}^\infty$ be a fixed $n$-plane, and let
$U \subset \grassmannian_n$ be the set of all $n$-planes $Y$ which project \emph{onto} $X_0$ under the
orthogonal projection $p\colon \mathbb{R}^\infty \to X_0$. This set $U$ is open since, for each
finite $k$, the intersection
\[
U_k=U\cap \grassmannian_n(\mathbb{R}^{n+k})
\]
is known to be an open set. Defining
\[
h\colon U\times X_0\to \pi^{-1}(U)
\]
as in Lemma~\ref{lem:05.02}, it follows from \ref{lem:05.02} that $h|_{U_k \times X_0}$ is continuous for each $k$.
Now Lemma \ref{lem:05.05} implies that $h$ itself is continuous.

As before, the identity $h^{-1}(Y,y) = (Y,py)$ implies that $h^{-1}$ is continuous. Thus $h$ is a homeomorphism. This completes the proof that
$\gamma^n$ is locally trivial.
\end{proof}

The following two theorems assert that this bundle $\gamma^n$ over $\grassmannian_n$ is a
``universal'' $\mathbb{R}^n$-bundle.\index{universal bundle}
\begin{theorem}\label{thm:05.06} Any $\mathbb{R}^n$-bundle $\xi$ over a paracompact base space admits a bundle map\index{bundle map} $\xi\to \gamma^n$.
\end{theorem}

Two bundle maps, $f, g\colon \xi\to \gamma^n$ are called \defemph{bundle-homotopic}\index{bundle homotopy} if there
exists a one-parameter family of bundle maps
\[
h_t\colon \xi\to \gamma^n, \quad 0\leq t\leq 1
\]
with $h_0 = f$, $h_1 = g$, such that $h$ is continuous as a function of both
variables. In other words the associated function
\[
h\colon \total(\xi)\times [0,1]\to \total(\gamma^n)
\]
must be continuous.
\begin{theorem}\label{thm:05.07} Any two bundle maps from an $\mathbb{R}^n$-bundle to $\gamma^n$
are bundle-homotopic.
\end{theorem}
\end{document} 