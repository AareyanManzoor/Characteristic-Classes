\documentclass[../main]{subfiles}

\begin{document}
Still assuming the existence of Stiefel-Whitney classes, this section will compute the $\mathrm{mod}\; 2$ cohomology of the infinite Grassmann manifold $\grassmannian_n=\grassmannian_n(\mathbb{R}^\infty)$, and will also prove a uniqueness theorem for Stiefel-Whitney classes. Recall that the canonical $n$-plane bundle over $\grassmannian_n$ is denoted by $\tautological^{n}$.
\begin{theorem}
\label{thm:07.01}
The cohomology ring $\homology^{\ast}(\grassmannian_n ; \, \mathbb{Z} / 2)$ is a polynomial algebra over $\mathbb{Z} / 2$ freely generated by the Stiefel-Whitney classes $\sw_{1}(\tautological^{n}), \dots, \sw_{n}(\tautological^{n})$.\index{Stiefel-Whitney class $\sw_i$}\index{cohomology!\indexline of $\grassmannian_n$}
\end{theorem}
To prove this result, we first show the following.
\begin{lemma}
\label{lem:07.02}
There are no polynomial relations among the $\sw_{i}(\tautological^{n})$.
\end{lemma}
\begin{proof}
Suppose that there is a relation of the form $p(\sw_{1}(\tautological^{n}), \dots,\sw_{n}(\tautological^{n}))=0$, where $p$ is a polynomial in $n$ variables with $\mathrm{mod}\; 2$ coefficients. By theorem \ref{thm:05.06}, for any $n$-plane bundle $\xi$ over a paracompact base space there exists a bundle map $g:\xi\varrightarrow{} \tautological^n$. Hence
\[\sw_{i}(\xi)=\xoverline{g}^{\ast}(\sw_{i}(\tautological^{n}))\]
where $\xoverline{g}$ is the map of base spaces induced by $g$. It follows that the cohomology classes $\sw_{i}(\xi)$ must satisfy the corresponding relation
\[p(\sw_{1}(\xi), \dots, \sw_{n}(\xi))=\xoverline{g}^{\ast} p(\sw_{1}(\tautological^{n}), \dots, \sw_{n}(\tautological^{n}))=0.\]
Thus to prove \ref{lem:07.02} it will suffice to find some $n$-plane bundle $\xi$ so that there are no polynomial relations among the classes $\sw_{1}(\xi), \dots, \sw_{n}(\xi)$. Consider the canonical line bundle $\tautological^{1}$ over the infinite projective space $\projective^\infty $. Recall from lemma \ref{lem:04.03} that $\homology^{\ast}(\projective^\infty ; \, \mathbb{Z} / 2)$ is a polynomial algebra over $\mathbb{Z} / 2$ with a single generator $a$ of dimension $1$ , and recall that $\sw(\tautological^{1})=1+a$. Forming the $n$-fold Cartesian product $X=\projective^\infty \times \dots \times \projective^\infty$ it follows that $\homology^{\ast}(X ; \, \mathbb{Z} / 2)$ is a polynomial algebra on $n$ generators $a_{1}, \dots, a_{n}$ of dimension $1 $. (Compare \ref{app:A}, theorem \ref{thm:A.6}; or \cite[p. 247]{spanier1981}.) Here  $a_{i}$ can be defined as the image $\pi_{i}^{\ast}(a)$ induced by the projection map $\pi_i:X\varrightarrow{} P$ to the $i$-th factor. Let $\xi$ be the $n$-fold cartesian product
\[\xi=\tautological^{1} \times \dots \times \tautological^{1} \cong(\pi_{1}^{\ast} \tautological^{1}) \oplus \dots \oplus(\pi_{n}^{\ast} \tautological^{1}).\]
Then $\xi$ is an $n$-plane bundle over $X=\projective^\infty \times \dots \times \projective^\infty$, and the total Stiefel-Whitney class
\[
\sw(\xi)=\sw(\tautological^{1}) \times \dots \times \sw(\tautological^{1})=\pi_{1}^{\ast}(\sw(\tautological^{1})) \dots \pi_{n}^{\ast}(\sw(\tautological^{1}))
\]
is equal to the $n$-fold product
\[
(1+a) \times \dots \times(1+a)=(1+a_{1})(1+a_{2}) \dots(1+a_{n})
\]
In other words
\begin{align*}
	&\sw_{1}(\xi)=a_{1}+a_{2}+\dots+a_{n} \\
	&\sw_{2}(\xi)=a_{1} a_{2}+a_{1} a_{3}+\dots+a_{1} a_{n}+\dots+a_{n-1} a_{n} \\
	&\sw_{n}(\xi)=a_{1} a_{2} \dots a_{n}
\end{align*}
and in general $\sw_{k}(\xi)$ is the $k$-th \defemph{elementary symmetric function} of $a_{1}, \dots, a_{n} $. It is proved in textbooks on algebra, that the $n$ elementary symmetric functions in $n$ indeterminates over a field do not satisfy any polynomial relations. (See for example \cite[pp 132-134]{lang1965algebra} or \cite[pp. 79,176]{waerden1970}.) Thus the classes $\sw_{1}(\xi), \dots, \sw_{n}(\xi)$ are algebraically independent over $\mathbb{Z} / 2$, and it follows as indicated above that $\sw_{1}(\tautological^{n}), \dots, \sw_{n}(\tautological^{n})$ are also algebraically independent.
\end{proof}
\begin{proof}[Proof of \ref{thm:07.01}]
We have shown that $\homology^{\ast}(\grassmannian_n)$, with $\mathrm{mod}\; 2$ coefficients, contains a polynomial algebra over $\mathbb{Z} / 2$ freely generated by $\sw_{1}(\tautological^{n}), \dots,$ $\sw_{n}(\tautological^{n})$. Using a counting argument, we will show that this sub-algebra actually coincides with $\homology^{\ast}(\grassmannian_n)$.
	
Recall from \ref{cor:06.07} that the number of $r$-cells in the CW-complex $\grassmannian_n$ is equal to the number of partitions\index{partition} of $r$ into at most $n$ integers. Hence the rank of $\homology^{r}(\grassmannian_n)$ over $\mathbb{Z} / 2$ is at most equal to this number of partitions. (In fact, if $C^{r}$ denotes the group of $\mathrm{mod}\; 2$ $r$-cochains for this CW-complex, and if $Z^{r} \supset B^{r}$ denote the corresponding cocycle and coboundary groups, then the number of $r$-cells equals\index{rank}
\[
\rank(C^{r}) \geq \rank(Z^{r}) \geq \rank(Z^{r} / B^{r})=\rank(\homology^{r}) \text{.)}
\]
On the other hand the number of distinct monomials of the form
\[\sw_{1}(\tautological^{n})^{r_{1}} \dots \sw_{n}(\tautological^{n})^{r_n}\]
in $\homology^{r}(\grassmannian_n)$ is also precisely equal to the number of partitions of $r$ into at most $n$ integers. For to each sequence $r_{1}, \dots, r_{n}$ of non-negative integers with
\[
r_{1}+2 r_{2}+\dots+nr_{n}=r
\]
we can associate the partition of $r$ which is obtained from the $n$-tuple
\[
r_{n},\, r_{n}+r_{n-1}, \dots,\, r_{n}+r_{n-1}+\dots+r_{1}
\]
by deleting any zeros which may occur; and conversely.
	
Since these monomials are known to be linearly independent $\mathrm{mod}\; 2$, it follows that the inequalities above must all actually be equalities: The module $\homology^{r}(\grassmannian_n)$ over $\mathbb{Z} / 2$ has rank equal to the number of partitions of $r$ into at most $n$ integers, and has a basis consisting of the various monomials $\sw_{1}(\tautological^{n})^{r_{1}} \dots \sw_{n}(\tautological^{n})^{r_{n}}$ of total dimension $r$.
\end{proof} 
It follows incidentally that the natural homomorphism\newline  $\xoverline{g}^\ast : \homology^\ast(\grassmannian_n)\varrightarrow{} \homology^\ast(\projective^\infty\times\dots\times\projective^\infty)$ maps $\homology^\ast(\grassmannian_n)$ isomorphically onto the algebra consisting of all polynomials in the indeterminates $a_1, \dots, a_n$ which are invariant under all permutations of these $n$ indeterminates.
\end{document}