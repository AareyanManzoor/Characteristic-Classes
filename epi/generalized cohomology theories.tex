\documentclass[../main]{subfiles}
\begin{document}
\section{Generalized Cohomology Theories}\index{cohomology!\indexline generalized}
So far we have discussed characteristic classes using ordinary cohomology theory, but using various exotic types of bundles. A quite different generalization arises if we use ordinary vector bundles, but generalize the cohomology. By definition, a \defemphi{generalized cohomology theory} is a functor \newline $(X, A) \mapsto {\mathcal H}^\ast(X, A)$ from pairs of spaces to graded additive groups which satisfies the first six Eilenberg-Steenrod axioms, but fails to satisfy the dimension axiom (the axiom that ${\mathcal H}^k(\text{point}) = 0$ for $k \ne 0$). Compare \cite{dyer1969cohomology}. The first and most important example of such a generalized cohomology theory is provided by \defemph{$\K$--theory}\index{K-theory@$\K$--theory}. 

\begin{definition}
For any compact space $X$ the additive group $\K^0(X)$ is defined by means of a presentation by generators and relations as follows. There is to be one generator $[\xi]$ for each isomorphism class of complex vector bundles $\xi$ over $X$ and one relation \[[\xi \oplus \eta] = [\xi] + [\eta]\] for each pair of complex vector bundles. For $m > 0$ the group $\K^{-m}(X)$ can be defined as the kernel of the natural surjection \[\K^0(\sphere^m \times X) \longrightarrow \K^0((\text{base point}) \times X).\] The tensor product operation for complex vector bundles gives rise to a product operation \[\K^{-m}(X) \otimes \K^{-n}(Y) \longrightarrow \K^{-m - n}(X \times Y).\] The \defemph{Bott periodicity theorem}\index{Bott periodicity} now asserts that the product with a standard generator in the group $\K^{-2}(\text{point}) \cong \mathbb Z$ yields an isomorphism \[\K^{-m}(X) \varrightarrow{\cong} \K^{-m - 2}(X).\] (This is closely related to the statement that the classifying space $\BU$ has the homotopy type of its own $2^{\text{nd}}$ loop space.) 
\end{definition}

The ring $\KO^\ast(X)$ is defined similarly, using real vector bundles in place of complex vector bundles. In this case there is a periodicity theorem \[\KO^{-m}(X) \varrightarrow{\cong} \KO^{-m - 8}(X).\] An illustrations of the powers of these methods, we refer the reader to \cite{atiyah1967},\cite{adams1960},\cite{adams1962},\newline \cite{adams1965},\cite{adams1966} and \cite{adams1972}. 

Similarly one can define the concept of a generalized homology theory\index{generalized homology theory}. One important example is provided by the \defemphi{stable homotopy groups} \[\pi_n^s(X) = \varinjlim \pi_{n + k}(\Sigma^k X),\] where $\Sigma^k X$ denotes the $k$--fold suspension of $X$. Another is provided by the \defemph{oriented bordism groups} $\Omega_n(X)$. (Compare \cite{connerfloyd}.) By definition two maps \[f_1 : M_1 \longrightarrow X, \quad f_2 : M_2 \longrightarrow X\] from smooth, compact, oriented $n$--manifolds to $X$ are called \defemphi{bordant} if there exists a smooth, compact, oriented manifold--with--boundary $N$\index{smooth manifold!\indexline with boundary} with \newline $\partial N = M_1 + (-M_2)$, and map $N \longrightarrow X$ extending $f_1$ and $f_2$. The bordism classes of such maps form a group $\Omega_n(X)$. Note that $\Omega_n(\text{point})$ is just the cobordism group\index{cobordism} $\Omega_n$ of Section~\ref{ch:17}. Each such generalized homology theory is associated with a corresponding generalized cohomology theory. See \cite{whitehead}. 

In order to study characteristic classes with values in a generalized cohomology theory such as $\K^\ast(B)$, one must first compute $\K^\ast$ of the appropriate classifying space\index{classifying space}. In the case of complex $\K$--theory, \cite{atiyah1961} establish an isomorphism between $\K^\ast(\operatorname{B}G)$ for a compact lie group $G$ and the completion of the representation ring\index{representation ring} of $G$. (See \cite{anderson1964} for the corresponding $\KO$--theory results.) 

Just as the \defemph{orientation}\index{oriented manifold} of a manifold using the classical homology theory $\homology_\ast(-; \mathbb Z)$ plays an important role in studying homology of manifolds, so the analogous $\K$--theory orientations play a basic role in studying the $\K$--theory of manifolds. (Compare \cite{shih1965}.) For example \cite{sullivan_2006} has proved the amazing result that a $\PL$--bundle is more or less the same thing as a spherical fibration\index{fibration} together with a $\KO$--orientation. 

For any $\K$--oriented bundle one can use the approach of Section~\ref{ch:8} and Section~\ref{ch:19} to define $\K$--theory characteristic classes, using appropriate $\K$--theory operations in place of the Steenrod operations. This idea was initially suggested by \cite{bott1962}, and was developed more extensively by \cite{adams1965}.  

As a typical illustration of the usefulness of these classes, consider the work of \cite{anderson1967} on spin cobordism. Suppose that one is given an oriented simply connected manifold $M$ with $\sw_2(M) = 0$. In order to test whether $M$ bounds an oriented manifold--with--boundary with $\sw_2 = 0$ one must check, not only that the Stiefel--Whitney numbers\index{Stiefel-Whitney number} (and Pontrjagin numbers\index{Pontrjagin number}) are zero, but also that all $\KO$--characteristic numbers are zero. 

If the cohomology theory is the one corresponding to complex bordism, \cite{connerfloyd} have introduced Chern--type classes. The algebra in this situation turns out to be particularly manageable so that rapid progress has been made by several people, notably \cite{Novikov_1967} (cf. \cite{adams1967}).
\end{document}