\documentclass[../main]{subfiles}
\begin{document}
\section{Smooth Manifolds with Additional Structure}
Instead of looking at non-differentiable manifolds, we can look at smooth manifolds which are provided with some additional structure. For example we can require that the ``structural group''\index{structural group} of the tangent bundle of our $n$-manifold (see \cite{steenrod1951} or \cite{husemoller}) should be some specified subgroup of the general linear group $\GL_n(\mathbb R)$ (or equivalently of the orthogonal group\index{orthogonal group} $\Orthogonal(n)$). One important example is provided by the unitary group\index{unitary group} $\U(n) \subset \Orthogonal(2 n)$. This leads to the study of almost complex manifolds and the closely related complex manifolds (Section \ref{ch:13}). Other examples are provided by the special unitary group $\SU(n) \subset \Orthogonal(2 n)$ and the compact symplectic group\index{symplectic group} $\Sp(n) \subset \Orthogonal(4 n)$. Similarly one can ``restrict'' the tangent bundle to the 2--fold covering $\Spin(n) \longrightarrow \SO(n)$. For a discussion of the cobordism\index{cobordism} theories associated with these various reductions, see \cite{stongcobordism1968}.\index{spinor group}

A different line of development is based on the definition of characteristic classes by means of differential forms\index{differential form}. (See Appendix \ref{app:C}.) These are particularly well adapted to the study of manifolds with some additional geometric structure, such as a foliation\index{foliation} or a Riemannian metric\index{Riemannian metric}. The vanishing of these classes in certain situations gives rise to new charac-. teristic classes, first studied from different points of view by \cite{chernsimmons} and \cite{godbillon-vey}. Some of these classes depend, for example, on the conformal structure of a Riemannian manifold. Some of the corresponding characteristic numbers can take on arbitrary real values (\cite{bott1972lefschetz}, \cite{baum}, and \cite{thurston}), showing the great richness of such structures. At this writing, this branch of the theory of characteristic classes is undergoing very rapid and vigorous development. A contemporary survey is given by \cite{bott-haefliger}.
\end{document}