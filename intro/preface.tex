\documentclass[../main]{subfiles}

\begin{document}
\chapter*{preface}
\addcontentsline{toc}{chapter}{Preface}

The text which follows is based mostly on lectures at Princeton
University in 1957. The senior author wishes to apologize for the delay
in publication.

The theory of characteristic classes began in the year 1935 with almost
simultaneous work by \defemph{Hassler Whitney}\index{Whitney, H.} in the United States and
\defemph{Eduard Stiefel}\index{Stiefel, E.} in Switzerland. Stiefel's thesis, written under the
direction of Heinz Hopf, introduced and studied certain ``characteristic''
homology classes determined by the tangent bundle of a smooth manifold.
Whitney, then at Harvard University, treated the case of an arbitrary sphere
bundle. Somewhat later he invented the language of cohomology theory,
hence the concept of a characteristic cohomology class, and proved the
basic product theorem.

In 1942 \defemph{Lev Pontrjagin}\index{Pontrjagin, L.} of Moscow University began to study the
homology of Grassmann manifolds, using a cell subdivision due to Charles
Ehresmann. This enabled him to construct important new characteristic
classes. (Pontrjagin's many contributions to mathematics are the more
remarkable in that he is totally blind, having lost his eyesight in an accident at the age of fourteen.)

In 1946 \defemph{Shing-Shen Chern}\index{Chern, S.S.}, recently arrived at the Institute for
Advanced Study from Kunming in southwestern China, defined characteristic classes for complex vector bundles. In fact he showed that the complex Grassmann manifolds have a cohomology structure which is much
easier to understand than that of the real Grassmann manifolds. This has
led to a great clarification of the theory of real characteristic classes.
We are happy to report that the four original creators of characteristic
class theory all remain mathematically active: Whitney at the Institute
for Advanced Study in Princeton, Stiefel as director of the Institute for
Applied Mathematics of the Federal Institute of Technology in Z\"urich,
Pontrjagin as director of the Steklov Institute in Moscow, and Chern at
the University of California in Berkeley. This book is dedicated to them.

\hfill -\defemph{John Milnor}

\hfill - \defemph{James Stasheff} 
\end{document}