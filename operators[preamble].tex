\newcommand{\dd}{\mathrm{d}}%for differentials and such
\DeclareMathOperator{\Hom}{Hom} %for Hom of bundles
\newcommand{\Vect}{\mathcal{V}} %category of finite dimensional vector spaces, see page 32 of book.
\DeclareMathOperator{\id}{id} %for identity function, ``identity'' in book.
\DeclareMathOperator{\GL}{GL} %for general linear group
\DeclareMathOperator{\SL}{SL} %special linear group
\DeclareMathOperator{\PSL}{PSL} %projective linear group
\DeclareMathOperator{\Orthogonal}{O}%orthogonal group
\DeclareMathOperator{\U}{U} %unitary group
\DeclareMathOperator{\SO}{SO} %special orthogonal
\DeclareMathOperator{\SU}{SU} %special unitary
\DeclareMathOperator{\Sp}{Sp} %symplectic group
\DeclareMathOperator{\Spin}{Spin} %spin group
\DeclareMathOperator{\grassmannian}{Gr} % for grassmannians, for example \grassmannian_n(V)
\DeclareMathOperator{\Exp}{Exp}
\DeclareMathOperator{\BSO}{BSO}
\DeclareMathOperator{\BO}{BO}
\DeclareMathOperator{\MSO}{MSO}
\DeclareMathOperator{\MO}{MO}
\DeclareMathOperator{\BU}{BU}
\DeclareMathOperator{\PL}{PL}
\DeclareMathOperator{\Min}{Min}
\DeclareMathOperator{\Max}{Max}
\DeclareMathOperator{\diag}{diag}
\DeclareMathOperator{\volume}{volume}
\DeclareMathOperator{\Orth}{O}

\DeclareMathOperator{\StiefelManifold}{V} %for stiefel manifolds, e.g \StiefelManifold_n(V).
%\DeclareMathOperator{\total}{E} %for total space of a bundle, e.g. \total(\xi)
%\DeclareMathOperator{\base}{B} %for base space of a bundle, e.g. \base(\xi)
\newcommand{\total}{E}
\newcommand{\base}{B}
\newcommand{\B}{B}
\DeclareMathOperator{\homology}{H} %for homology/cohomology, \homology^i(M,\mathbb{Z}/2) for example
\newcommand{\generalizedHomology}{\mathcal{H}} %generalized homology theories.
\DeclareMathOperator{\K}{K} %K theory
\DeclareMathOperator{\KO}{KO} %Real K theory
\newcommand{\orientedCobordism}{\Omega} %to notate oriented cobordism groups, \cobordism_n for example.
\newcommand{\unorientedCobordism}{\mathfrak{N}} %for unoriented cobordism groups.
\newcommand{\disk}{\mathbb{D}} %for disk, \disk^p \subset \mathbb{R}^n for example
\newcommand{\halfspace}{\mathbb{H}} %for upper half spaces H^k in book, \halfspace^k for example.
\DeclareMathOperator{\steenrod}{Sq} %for the steenrod squares.
\newcommand{\reducedSteenrod}{\mathcal{P}} %for reduced steenrod powers
\DeclareMathOperator{\rank}{rank} %for rank of e.g. abelian groups
\DeclareMathOperator{\thom}{Th} %for thom space of a bundle, T(\xi) in the book.
\newcommand{\Ab}{\mathcal{C}}
\newcommand{\steenrodReduced}{\mathcal{P}}
\DeclareMathOperator{\Kernel}{Kernel}
\DeclareMathOperator{\Image}{Image}
\DeclareMathOperator{\Sq}{Sq}

\DeclareMathOperator{\trace}{trace}

\newcommand{\restr}{\big{|}} %function restrictions
\newcommand{\ip}[2]{\langle#1,#2\rangle} %for inner product <1,2>
\newcommand{\multilinecomment}[1]{}


\newcommand{\trivialbundle}{\varepsilon} %trivial bundle,\varepsilon in book.
\newcommand{\tangentspace}[2]{\mathbf{T}_{#2} #1} %tangentspace{M}{x} = tangent space of M at x
\newcommand{\tangentspacemap}[2]{\dd #1_{#2}} %for maps between tangent space :\tangentspacemap{f}{x}= the differential of f at x.
\newcommand{\differential}[2]{\tangentspacemap{#1}{#2}}
\newcommand{\tangentbundlemap}[1]{\dd #1} %for maps between tangent bundles, \tangentbundlemap{f} = df:TM->TN
\newcommand{\tangentTS}[1]{\mathbf{T}#1} %total space of the tangent bundle, DM in the book. 

\newcommand{\tangentbundle}[1]{\tau_{#1}} %\tangentbundle{M} = tangent bundle of M (\tau_M) in the book
\newcommand{\normalbundle}[1]{\nu_{#1}} %normal bundle corrosponding to immersion #1, \nu_{#1} in book.


\newcommand{\projective}{\mathbb{P}} %projective space, P^n in book, do \projective^n
\newcommand{\tautological}{\gamma} %tautological/cannonical or universal bundles over grassmananians/projective space, \gamma in the book.


\newcommand{\diagonal}{\Delta} %(for diagonal embeddings)

\DeclareMathOperator{\sw}{w} %stiefel-Whitney class, do \sw_i(\xi) or \sw(\xi) depending on context. w in text.
\DeclareMathOperator{\eulerclass}{e} %eulerclass, do \eulerclass_i(\xi) or \eulerclass(\xi)
\newcommand{\obstruction}{\mathfrak{o}} %for obstruction classes.
\DeclareMathOperator{\chernclass}{c} %for chern class, \chernclass_i(\xi) or \chernclass(\xi)
\DeclareMathOperator{\chernchar}{ch}
\DeclareMathOperator{\pontrjaginclass}{p} %for pontrjagin classes
\DeclareMathOperator{\multclass}{k} %for multiplicative classes
\DeclareMathOperator{\steenrodReducedclass}{q} %see page 228

\newcommand{\vectCat}{\mathbf{Vect}}

\newcommand{\defemph}{\textbf} %to emphasize text
\newcommand{\complexstructure}{\mathbf{J}} %for complex structures on bundles, given as J: E(\xi) -> E(\xi) in text.

\newcommand{\defemphi}[1]{\defemph{#1}\index{#1}}

%%use this to overset/underset things to arrows, and in general use \varrightarrow{} for maps 

\makeatletter

\def\@@varrightarrow#1#2#3{\begingroup%
\setbox0=\hbox{$#1\xrightarrow[#3]{#2}$}%
\setbox1=\hbox{$#1\longrightarrow$}%
\ifdim\wd0<\wd1 \mathrel{\mathop{\longrightarrow}\limits^{#2}_{#3}}
\else \xrightarrow[#3]{#2} \fi\endgroup}

\def\@varrightarrow#1#2{\@@varrightarrow#1#2}

% \varrightarrow[under material]{over material} puts material over and under a right arrow.
% It automatically prints the longer option between an xrightarrow and a longrightarrow
\newcommand\varrightarrow[2][]{\mathpalette\@varrightarrow{{#2}{#1}}}
%%%%%

\newcommand{\xoverline}[1]{\mskip.5\thinmuskip\overline{\mskip-.5\thinmuskip {#1} \mskip-.5\thinmuskip}\mskip.5\thinmuskip}
\newcommand{\xhat}[1]{\mskip.5\thinmuskip\widehat{\mskip-.5\thinmuskip {#1} \mskip-.5\thinmuskip}\mskip.5\thinmuskip}
\newcommand{\xtilde}[1]{\mskip.5\thinmuskip\widetilde{\mskip-.5\thinmuskip {#1} \mskip-.5\thinmuskip}\mskip.5\thinmuskip}

\newcommand{\bR}{\mathbb R}
\newcommand{\bC}{\mathbb C}
\newcommand{\inv}{^{-1}}

\newcommand{\br}[1]{\left(#1\right)}
\newcommand{\sbr}[1]{\left[#1\right]}
\newcommand{\brc}[1]{\left\{#1\right\}}

\newcommand{\abfin}{\mathbf{Ab}_{<\infty}}

%%%%redundant commands, feel free to ignore
\newcommand{\sphere}{S} %no need for this, we are using S, but we will keep this
\newcommand{\Sphere}{S} %no need for this, we are using S, but we will keep this
\newcommand{\tangentBundle}[1]{\tau_{#1}} %\tangentbundle{M} = tangent bundle of M (\tau_M) in the book
\newcommand{\map}[3]{#1:#2\varrightarrow{} #3}
