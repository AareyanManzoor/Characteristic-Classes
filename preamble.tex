\usepackage{makeidx}
\makeindex
\usepackage[utf8]{inputenc}
\usepackage[T1]{fontenc}

\usepackage{amsmath}
\usepackage{amsfonts}
\usepackage{amsthm}
\usepackage{mathtools}
\usepackage{amssymb}
\usepackage[shortlabels]{enumitem}
\usepackage{indentfirst}
\usepackage{mathrsfs}
\usepackage{ragged2e}
\usepackage[style=alphabetic-verb]{biblatex}
\addbibresource{references.bib}
\usepackage{tikz-cd,tikz}
    \usetikzlibrary{hobby, calc, intersections, decorations.markings, decorations.pathreplacing} %libraries
    \tikzset{>=latex} %sets -> to render as -latex arrow tip
                      %https://tex.stackexchange.com/questions/54796/how-to-set-default-style-for-arrow-tips-in-tikz
\input{insbox} %used for a floating diagram in sec 13 --deri https://tex.stackexchange.com/a/486033
%\usepackage{quiver}
\usepackage{subfiles}

\usepackage[b5paper]{geometry} %page size, same as Blue Book



\newcommand{\plscite}[1]{{\color{purple} #1}} %Changed from green to purple to stop the eye massacre

\usepackage{chngcntr}
\usepackage{ifthen}

\interfootnotelinepenalty=10000

\theoremstyle{definition}
\newtheorem{theorem}{Theorem}[chapter]
\newtheorem{proposition}{Proposition}
\newtheorem{preposition}{Preposition}
\newtheorem{property}{Property}
\newtheorem{lemma}[theorem]{Lemma}
\newtheorem{corollary}[theorem]{Corollary}
\newtheorem{example}{Example}
\newtheorem{axiom}{Axiom}
\newtheorem{remark}{Remark}
\newtheorem{condition}{Condition}
\newtheorem{problem}{Problem}[chapter]

%problems are counted a bit differently

\renewcommand{\theproblem}{\thechapter-\Alph{problem}}


%numberless env's:
\newtheorem*{definition}{Definition}
\newtheorem*{examples}{Example}
\newtheorem*{exercise}{Exercise}
\newtheorem*{notes}{Notes}
\newtheorem*{note}{Note}
\newtheorem*{hint}{Hint}
\newtheorem*{proposition*}{Proposition}
\newtheorem*{theorem*}{Theorem}
\newtheorem*{lemma*}{Lemma}
\newtheorem*{corollary*}{Corollary}
\newtheorem*{remark*}{Remark}
\newtheorem*{remarks*}{Remarks}
\newtheorem*{explanation}{Explanation}
\newtheorem*{warning}{Warnings}
\newtheorem*{problems}{Problems}
\newtheorem*{assertion*}{Assertion}
\newtheorem*{hypo}{Hypothesis}



% this is just for chapter 1 and 3 where there's a small paragraph after the chapter title
\renewenvironment{quote}{
  \noindent     % No indent of first character.
  \topsep=0pt   % No additional separator before.
  \parskip=0pt  % No paragraph separator.
  \interlinepenalty=10000  % Prevent a page break to occur in a quote block.
  \leftskip=0pt 
  \rightskip=0pt 
  \parfillskip=0pt
  \itshape  % Make quotations in italic.
  \list{}{
    \leftmargin 5em   % this is the adjusting screw
    \rightmargin 5em  % this is the adjusting screw
  }
  \item
  \relax
  }
  {\endlist}


\newenvironment{implyenumerate}
    {\begin{enumerate}[align=left,leftmargin=!,labelindent=5pt,itemindent=-15pt]
    }
    {
    \end{enumerate}
    }
  



%solution for custom numbering in theorems and lemmata:
\newtheorem{innercustomgeneric}{\customgenericname}
\providecommand{\customgenericname}{}
\newcommand{\newcustomtheorem}[2]{%
  \newenvironment{#1}[1]
  {%
   \renewcommand\customgenericname{#2}%
   \renewcommand\theinnercustomgeneric{##1}%
   \innercustomgeneric
  }
  {\endinnercustomgeneric}
}
\newcustomtheorem{customthm}{Theorem}
\newcustomtheorem{customlemma}{Lemma}
\newcustomtheorem{customcor}{Corollary}
\newcustomtheorem{customprop}{Proposition}
\newcustomtheorem{customexample}{Examples}
\newcustomtheorem{customremark}{Remark}
\newcustomtheorem{customdef}{Definition}

\usepackage{setspace}
\setstretch{1.25}

\renewcommand\thesection{\thechapter.\arabic{section}}

\usepackage{titlesec}
\titleformat{\chapter}[hang]
{\scshape\fillast\LARGE\bfseries}
{\thechapter .}
{0.5em}
{}

\titleformat*{\section}{\large\bfseries}

\usepackage{fancyhdr}
\fancyhf{}
\renewcommand{\chaptermark}[1]{\markboth{#1}{}}
\fancyhead[LE]{\scshape Chapter \thechapter: \leftmark }
\renewcommand{\sectionmark}[1]{ \markright{#1}{} }
\fancyhead[RO]{
\ifthenelse{\value{section}= 0}{\scshape Chapter \thechapter: \leftmark}{\scshape Section \thesection: \rightmark}
 }
\renewcommand{\headrulewidth}{0.3pt}
\renewcommand{\footrulewidth}{\iffootnote{0pt}{0.3pt}}

\fancyfoot[C]{\thepage}

\titleformat{\part}[display]
{\centering\scshape\huge}
{Part \thepart}
{0.5em}
{}
\newcommand{\indexline}{---\ }


\usepackage{adjustbox}

% https://tex.stackexchange.com/questions/253910/reference-to-enumerate-item-with-manually-set-label
% this is just for chapter 34 that dumb reference
\newcounter{dummy}
\makeatletter
\newcommand\myitem[1][]{\item[#1]\refstepcounter{dummy}\def\@currentlabel{#1}}
\makeatother


%%%to box stuff in aligned
\usepackage[version=4]{mhchem}
\usepackage{siunitx}

\newcounter{markeq}
\setcounter{markeq}{0}

\newcommand{\pstrut}[1]{\vrule height0pt depth0pt width0pt #1 \fboxsep}
\newcommand*\bmarkeq{\stepcounter{markeq}%
  \tikz[remember picture]\node(startframe-\themarkeq){\pstrut{height}};%
  \kern\fboxsep}
\newcommand*\emarkeq{\kern\fboxsep
  \begin{tikzpicture}[remember picture,overlay]
    \node (endframe-\themarkeq){\pstrut{depth}};
    \draw[,red,opacity=0.8] (startframe-\themarkeq.north) 
      rectangle (endframe-\themarkeq.south);
  \end{tikzpicture}%
}
\usepackage{tablefootnote}

%%%hyperref and Friends go at the end
\usepackage[bookmarks=true, colorlinks=true, linkcolor=blue, citecolor= red,hypertexnames=false,pdftex,
            pdfauthor={Milnor, Stasheff},
            pdftitle={Characteristic Classes},
            pdfsubject={Differential Topology},
            pdfkeywords={Milnor, Stasheff, Characteristic Classes, TeXromancers},
            pdfproducer={Latex with hyperref},
            pdfcreator={pdflatexl}]{hyperref}
\usepackage[hypcap=true]{caption}
\newcommand{\parref}[1]{\hyperref[#1]{#1}} %to ref paragraphs.
\usepackage{xurl}

% Needed for drawing
\usepackage{import}
\usepackage{xifthen}
\usepackage{pdfpages}
\usepackage{transparent}

\newcommand{\incfig}[1]{%
    \def\svgwidth{\columnwidth}
    \import{./figures/}{#1.pdf_tex}
}