
%\usepackage[hypcap]{caption}
%\usepackage{titlesec}
%\renewcommand{\sectionbreak}{\clearpage}
\usepackage[hypcap=true]{subcaption}

\captionsetup[subfigure]{singlelinecheck=on}
\usepackage{amsthm}
\usepackage{amsmath,amsfonts}
\usepackage{amssymb}
%\usepackage[scr=pxtx]{mathalfa}
%\usepackage{diffcoeff}
%\usepackage{bbold}
%\usepackage[sans]{dsfont}
%\usepackage{fourier}

\usepackage[Symbol]{upgreek}
\usepackage{graphicx}
\usepackage[colorlinks,linkcolor=blue,citecolor=green]
{hyperref}
\usepackage[nameinlink]{cleveref}
%\usepackage[figure]{hypcap}%[all]

\newtheoremstyle{sans}{3pt}{3pt}{}{}{\bfseries\sffamily}{.}{.5em}{}
\newtheoremstyle{sansit}{3pt}{3pt}{\itshape}{}{\bfseries\sffamily}{.}{.5em}{}

 \AtBeginDocument{%
 	\crefname{equation}{equation}{equations}%
 	\crefname{figure}{Figure}{Figures}%
 	\crefname{assertion}{Assertion}{Assertions}%
 	\crefname{theorem}{Theorem}{Theorems}%
 	\crefname{lemma}{Lemma}{Lemmas}%
 	\crefname{corollary}{Corollary}{Corollaries}%
 	\crefname{proposition}{Proposition}{Propositions}%
 	\crefname{definition}{Definition}{Definitions}%
 	\crefname{hypo}{Preliminary Hypothesis}{preliminary hypothesis}%
 	\crefname{example}{Example}{Examples}%
 	\crefname{remark}{Remark}{Remarks}%
 	\crefname{problem}{Problem}{Problems}%
 	\crefname{property}{Property}{Properties}%
 	\crefname{section}{\S}{\S\S}%
 }

\theoremstyle{sansit}
\newtheorem{theorem}{Theorem}[chapter]%
\newtheorem{corollary}[theorem]{Corollary}%
\newtheorem{lemma}[theorem]{Lemma}%
\newtheorem{proposition}[theorem]{Proposition}%
\newtheorem{assumption}[theorem]{Assumption}%
\newtheorem{conjecture}[theorem]{\conjecturename}%
\theoremstyle{sans}
\newtheorem{definition}[theorem]{Definition}%
\newtheorem*{remark*}{Remark}%
\newtheorem*{note}{Note}%
\newtheorem{remark}{Remark}%
\newtheorem{example}{Example}%
\newtheorem{exercise}[theorem]{Exercise}%

\hypersetup{
	pdftitle={CHARACTERISTIC CLASSES},
	pdfauthor={John W. Milnor},
	pdfsubject={Lecture Note},
	pdfkeywords={CHARACTERISTIC CLASSES; Stiefel-Whitney Classes; Chern Classes; Stiefel-Whitney Classes}
}
%\usepackage{draftwatermark}%,
%\SetWatermarkScale{.7}
%\SetWatermarkText{Draft Version}
%\usepackage{natbib}


\usepackage{lmodern}
%\usepackage{bookmark}
\usepackage{enumitem}
%\setlist[enumerate]{labelindent=\parindent, leftmargin=*,
%	widest=IV, align=left}%,leftmargin=1.5cm
\usepackage{tikz-cd,tikz}
\usetikzlibrary{arrows,calc,decorations.markings,positioning,decorations.pathreplacing,intersections,patterns,matrix}

%%%%%%%%%%%%%%%%%%%%%%%%%%%%%%%%%%%%
\newcommand{\set}[1]{\setaux#1\relax}
\def\setaux#1#2#3\relax{%
	\{ {#1}#2 1,
	\ifnum\pdfstrcmp{#3}{3}=0
	{#1}#2 2
	\else
	\dots
	\fi
	, {#1}#2{#3} \}
}

\newcommand{\ntuple}[1]{\ntupleaux#1\relax}
\def\ntupleaux#1#2#3\relax{%
	( {#1}#2 1,
	\ifnum\pdfstrcmp{#3}{3}=0
	{#1}#2 2
	\else
	\dots
	\fi
	, {#1}#2{#3} )
}

\newcommand{\base}[1]{\baseaux#1\relax}
\def\baseaux#1\relax{%
	\pd{}{{#1}^1}|_{p},
	\dots
	, \pd{}{{#1}^n}|_{p}
}
\newcommand{\slbase}[1]{\baseaux#1\relax}
\def\baseaux#1\relax{%
	\slpd{}{{#1}^1}|_{p},
	\dots
	, \slpd{}{{#1}^n}|_{p}
}

\newcommand{\coord}[1]{\coordaux#1\relax}
\def\coordaux#1#2#3#4\relax
%%%%%%%%%%%%%%%%%%%%%%%%%%%%%%%%%%%%

\newcommand{\B}{\mathcal{B}}
\let\rand\partial
\newcommand{\R}{\boldsymbol{\mathbb{R}}}
\newcommand{\Z}{\mathbb{Z}}
\newcommand{\Cx}{\mathbb{C}}
\newcommand{\F}{\mathbb{F}}
\newcommand\Sphere[1]{\mathbb{S}^{#1}}
\newcommand\cp[1]{\Cx\boldsymbol{\mathrm{P}}^{#1}}
\newcommand\rp[1]{\R\boldsymbol{\mathrm{P}}^{#1}}

\DeclareMathAlphabet{\mathlinlib}{T1}{LinuxLibertineT-TLF}{m}{n}
%\SetMathAlphabet{\mathlinlib}{bold}{T1}{LinuxLibertineT-TLF}{b}{n}
\newcommand{\Q}{\mathlinlib{Q}}
\DeclareMathAlphabet{\oldmathcal}{OMS}{cmsy}{m}{n}

\newcommand{\rank}{\mathrm{Rank}\,}
\newcommand{\Hom}{\mathsf{Hom}}
\newcommand{\smooth}{C^\infty}
\newcommand{\id}{\mathbf{Id}}
\newcommand{\supp}{\mathrm{Supp}\,}
\newcommand{\grad}{\mathsf{grad}\,}
\newcommand{\dif}{\mathsf{d}}
\newcommand{\homology}{\mathcal{H}}
\newcommand{\norm}[1]{\left\|#1\right\|}

\newcommand\pd[2]{\frac{\partial #1}{\partial #2}}
\newcommand\Pd[2]{\frac{\mathsf{D} #1}{\partial #2}}
\newcommand\dpd[2]{\dfrac{\partial #1}{\partial #2}}
\newcommand\slpd[2]{\partial #1/ \partial #2}
\newcommand\spd[3]{\frac{\partial^2 #1}{\partial #2 \partial #3}}
\newcommand\dspd[3]{\dfrac{\partial^2 #1}{\partial #2 \partial #3}}
\newcommand\slspd[3]{\partial^2 #1/\partial #2 \partial #3}

\newcommand\od[2]{\frac{\dif #1}{\dif #2}}
\newcommand\dod[2]{\dfrac{\dif #1}{\dif #2}}
\newcommand\slod[2]{\dif #1/ \dif #2}
\newcommand\sod[2]{\frac{\dif^2 #1}{\dif {#2}^2}}
\newcommand\Od[2]{\frac{\mathsf{D} #1}{\dif #2}}
\newcommand\dOd[2]{\dfrac{\mathsf{D} #1}{\dif #2}}
\newcommand\ip[1]{\left\langle #1\right\rangle}
\newcommand\inv{^{-1}}
\renewcommand\lambda{\uplambda}
\renewcommand\chi{\upchi}
\font\ursymbol=psyr at 11pt % You also use other font sizes.
\def\urpartial{\mbox{\ursymbol\char"B6}}
\renewcommand\partial{\urpartial}
\newcommand\grass[2]{\def\temp{#2}\ifx\temp\empty
	\mathsf{Gr}_{#1}
	\else
	\mathsf{Gr}_{#1}(\R^{#2})
	\fi}
\newcommand\stiefel[3]{\def\temp{#2}\ifx\temp\empty
	\mathsf{V}_{#1}(\R^{#3})
	\else
	\mathsf{V}_{#1}^{#2}(\R^{#3})
	\fi}
\newcommand{\sq}{\mathsf{Sq}}

\newcommand\disk[1]{\mathbb{D}^{#1}}
\newcommand\odisk{\mathrm{OD}}
\newcommand{\bdotp}{\operatorname{\boldsymbol{\cdot}}}

\newcommand{\map}[3]{#1\mathpunct{:}#2\mathchoice{\longrightarrow}{\to}{\to}{\to}#3}


\theoremstyle{sansit}
\newtheorem{assertion}{Assertion}%[section]
\newtheorem{property}[theorem]{Property}
\newtheorem*{assertion*}{Assertion}

\theoremstyle{sans}
\newtheorem{problem}{Problem}[chapter]
\renewcommand{\theproblem}{\thechapter-\Alph{problem}}

\theoremstyle{definition}
\newtheorem{application}{Application}[chapter]

\renewcommand{\thesection}{\arabic{section}}

\newcommand*\embraced[2]{\tikz[baseline=(char.base)]{
		\node[inner sep=-1.5pt] (char) {\color{#1}\ttfamily \char`\{\textcolor{#1}{#2}\color{#1}\ttfamily \char`\}};}}
\newcommand*\oline[1]{%
	\kern0.1em            % Counteract the inner kern
	\vbox{%
		\hrule height 0.5pt % Line above with certain width
		\kern0.4ex          % Distance between line and content
		\hbox{%
			\kern-0.1em       % Shorten the content on the left
			$#1$%             % The content, typeset math mode
			\kern-0.1em       % Shorten the content on the right
		}% end of hbox
	}% end of vbox
	\kern0.1em            % Counteract the inner kern
}



%%%%%%%%%%%%%%%%%%%%%%%%%%%%%%%%%%%%%%
\def\truncdiv#1#2{((#1-(#2-1)/2)/#2)}
\def\moduloop#1#2{(#1-\truncdiv{#1}{#2}*#2)}
\def\modulo#1#2{\number\numexpr\moduloop{#1}{#2}\relax}
% \binomialb macro from https://tex.stackexchange.com/a/161863/4686
% expandably computes binomial coefficients with \numexpr

% START OF CODE
\catcode`_ 11

\def\binomialb #1#2{\romannumeral0\expandafter
	\binomialb_a\the\numexpr #1\expandafter.\the\numexpr #2.}

\def\binomialb_a #1.#2.{\expandafter\binomialb_b\the\numexpr #1-#2.#2.}

\def\binomialb_b #1.#2.{\ifnum #1<#2 \expandafter\binomialb_ca
	\else   \expandafter\binomialb_cb
	\fi {#1}{#2}}

\def\binomialb_ca #1{\ifnum#1=0 \expandafter \binomialb_one\else 
	\expandafter \binomialb_d\fi {#1}}

\def\binomialb_cb #1#2{\ifnum #2=0 \expandafter\binomialb_one\else
	\expandafter\binomialb_d\fi {#2}{#1}}

\def\binomialb_one #1#2{ 1}

\def\binomialb_d #1#2{\expandafter\binomialb_e \the\numexpr #2+1.#1!}

% n-k+1.k! -> u=n-k+2.v=2.w=n-k+1.k!
\def\binomialb_e #1.{\expandafter\binomialb_f \the\numexpr #1+1.2.#1.}

% u.v.w.k!
\def\binomialb_f #1.#2.#3.#4!%
{\ifnum #2>#4 \binomialb_end\fi
	\expandafter\binomialb_f
	\the\numexpr #1+1\expandafter.%
	\the\numexpr #2+1\expandafter.%
	\the\numexpr #1*#3/#2.#4!}

\def\binomialb_end #1*#2/#3!{\fi\space #2}
\catcode`_ 8
% END OR \binomialb code
%%%%%%%%%%%%%%%%%%%%%%%%%%%%%%%%%%%%%%



%\usepackage{mathenv,nath}
\usepackage{braket}%,mathtools
\usepackage{empheq}

\renewcommand{\sectionmark}[1]{}%prevent rewriting \markright 
\renewcommand{\chaptermark}[1]{}%prevent rewriting \markright 

\renewcommand{\sectionmark}[1]{%
	\markright{#1}}

\copypagestyle{headingsnobook}{headings}
\makeevenhead{headingsnobook}{\normalfont\fontsize{10}{10pt}\selectfont\embraced{red}{\textcolor{black}{\thepage}}}{\timedate}{\normalfont\fontsize{10}{10pt}\selectfont\sffamily %Lectures on the h-Cobordism Theorem}
	\embraced{gray}{\thesection}\ \leftmark}
%
\makeoddhead{headingsnobook}{\normalfont\fontsize{10}{10pt}\selectfont\sffamily \embraced{gray}{\thesection}\  \leftmark}{\timedate}{\normalfont\fontsize{10}{10pt}\selectfont\embraced{red}{\textcolor{black}{\thepage}}}
\renewcommand{\prefacename}{Introduction}


\usepackage{refcount}
\newcommand{\pagedifference}[2]{%
	\number\numexpr\getpagerefnumber{#2}-\getpagerefnumber{#1}\relax}

\usepackage{makeidx}

\makeindex
