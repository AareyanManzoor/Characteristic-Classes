%\clearpage

%\addtocontents{toc}{\vspace*{-6pt}}
%\begin{theforeword}
%
%\end{theforeword}

\cleardoublepage

\addtocontents{toc}{\vspace*{-6pt}}
\renewcommand{\prefacename}{Preface to the First Edition}
\begin{thepreface}
	The text which follows is based mostly on lectures at Princeton
	University in $1957$. The senior author wishes to apologize for the delay
	in publication.
	
	The theory of characteristic classes began in the year $1935$ with almost
	simultaneous work by HASSLER WHITNEY in the United States and
	EDUARD STIEFEL in Switzerland. Stiefel's thesis, written under the
	direction of Heinz Hopf, introduced and studied certain ``characteristic"
	homology classes determined by the tangent bundle of a smooth manifold.
	Whitney, then at Harvard University, treated the case of an arbitrary sphere
	bundle. Somewhat later he invented the language of cohomology theory,
	hence the concept of a characteristic cohomology class, and proved the
	basic product theorem.
	
	In $1942$ LEV PONTRJAGIN of Moscow University began to study the
	homology of Grassmann manifolds, using a cell subdivision due to Charles
	Ehresmann. This enabled him to construct important new characteristic
	classes. (Pontrjagin's many contributions to mathematics are the more
	remarkable in that he is totally blind, having lost his eyesight in an 
	accident at the age of fourteen.)
	
	In $1946$ SHING-SHEN CHERN, recently arrived at the Institute for
	Advanced Study from Kunming in southwestern China, defined 
	characteristic classes for complex vector bundles. In fact he showed that the 
	complex Grassmann manifolds have a cohomology structure which is much
	easier to understand than that of the real Grassmann manifolds. This has
	led to a great clarification of the theory of real characteristic classes.
	
	We are happy to report that the four original creators of characteristic
	class theory all remain mathematically active: Whitney at the Institute
	for Advanced Study in Princeton, Stiefel as director of the Institute for
	Applied Mathematics of the Federal Institute of Technology in Zurich,
	Pontrjagin as director of the Steklov Institute in Moscow, and Chern at
	the University of California in Berkeley. This book is dedicated to them.
	
	

\begin{flushright}
	JOHN MILNOR\\
	JAMES STASHEFF\\
	Princeton University.
\end{flushright}
\end{thepreface}

%\cleardoublepage
%
%\addtocontents{toc}{\vspace*{-6pt}}
%\begin{theacknowledgment}
%
%\end{theacknowledgment}