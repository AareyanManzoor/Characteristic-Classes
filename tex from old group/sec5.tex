\chapter{Grassmann Manifolds and Universal Bundles}

\section*{title}\label{sec5.8}



\subsection*{Infinite Grassmann Manifolds}

A similar argument applies if the base space $B$ is paracompact and finite dimensional. (Compare \cref{prob-5-E}.) However in order to take care of bundles over more exotic base spaces it is necessary to allow the dimension of $\R^{n+k}$ to tend to infinity, thus yielding an infinite Grassmann "manifold" $g_{n}(r^{\infty})$.

Let $\R^{\infty}$ denote the vector space consisting of those infinite sequences
\[
x=(x_{1}, x_{2}, x_{3}, \dots\ )
\]
of real numbers for which all but a finite number of the $X_{i}$ are zero. (Thus $\R^{\infty}$ is much smaller than the infinite coordinate spaces utilized in \cref{ch-1}.)

For fixed $k$, the subspace consisting of all

\[
x=(x_{1}, x_{2}, \dots, x_{k}, 0,0, \dots\ )
\]

will be identified with the coordinate space $\R^{k}$. Thus $\R^{1} \subset \R^{2} \subset \R^{3} \subset \dots$ with union $\R^{\infty}$.

\begin{definition}\label{def:5-2}
	The infinite Grassmann manifold
	\[
	\mathsf{G}_{n}=\grass{n}{\infty}
	\]
	is the set of all $n$-dimensional linear sub-spaces of $\R^{\infty}$, topologized as the direct limit\footnote{It is customary in algebraic topology to call this the ``weak topology," a weak topology being one with many open sets. This usage is unfortunate since analysts use the term weak topology with precisely the opposite meaning. On the other hand the terms ``fine topology" or ``large topology" or ``Whitehead topology" are certainly acceptable.} of the sequence
	\[
	\grass{n}{n} \subset \grass{n}{n+1} \subset \grass{n}{n+2} \subset \dots\ .
	\]
	In other words, a subset of $\mathsf{G}_{n}$ is open (or closed) if and only if its intersection with $\grass{n}{n+k}$ is open (or closed) as a subset of $\grass{n}{n+k}$ for each $k$. This makes sense since $\grass{n}{\infty}$ is equal to the union of the subsets $\grass{n}{n+k}$.
	
	As a special case, the infinite projective space $\rp{\infty}=\grass{1}{\infty}$ is equal to the direct limit of the sequence $\rp{1} \subset \rp{2} \subset \rp{3} \subset \dots\ $.
	
	Similarly $\R^{\infty}$ itself can be topologized as the direct limit of the sequence $\R^{1} \subset \R^{2} \subset \dots\ $.
\end{definition}

\subsection*{The Universal Bundle $\protect\boldsymbol{\gamma}^{n}$}

A canonical bundle $\gamma^{n}$ over $\grass{n}{}$ is constructed, just as in the finite dimensional case, as follows. Let
\[
E(\gamma^{n}) \subset \grass{n}{} \times \R^{\infty}
\]
be the set of all pairs
\[
 (n\text{-plane in } \R^{\infty}, \text { vector in that } n\text{-plane)},
\]
topologized as a subset of the Cartesian product. Define $\map{\pi}{E(\gamma^{n})}{\grass{n}{}}$ by $\pi(X, x)=X$, and define the vector space structures in the fibers as before.

\begin{lemma}\label{lem-5-4}
	This vector bundle $\gamma^{n}$ satisfies the local triviality condition.
\end{lemma}

The proof will be essentially the same as that of \cref{lem-5-2}. However the following technical lemma will be needed. (Compare \cite[\S~18.5]{82}.)

\begin{lemma}\label{lem-5-5}
	Let $A_{1} \subset A_{2} \subset \dots$ and ${B}_{1} \subset {B}_{2} \subset \dots$ be sequences of locally compact spaces with direct limits A and B respectively. Then the Cartesian product topology on $A \times {B}$ coincides with the direct limit topology which is associated with the sequence $A_{1} \times {B}_{1} \subset A_{2} \times {B}_{2} \subset \dots\ .$
\end{lemma}
\begin{proof}
	Let $W$ be open in the direct limit topology, and let $(a, b)$ be any point of $W $. Suppose that $(a, b) \in A_{i} \times B_{i} $. Choose a compact neighborhood $k_{i}$ of a in $A_{i}$ and a compact neighborhood ${L}_{i}$ of ${B}$ in ${B}_{i}$ so that $K_{i} \times {L}_{i} \subset W$. It is now possible (with some effort) to choose compact neighborhoods $K_{i+1}$ of $K_{i}$ in $A_{i+1}$ and $L_{i+1}$ of $L_{i}$ in $B_{i+1}$ so that $K_{i+1} \times {L}_{i+1} \subset W$. Continue by induction, constructing neighborhoods $K_{i} \subset K_{i+1} \subset K_{i+2} \subset \dots$ with union $U$ and ${L}_{i} \subset {L}_{i+1} \subset \dots$ with union $V$. Then $U$ and $V$ are open sets, and
	\[
	(a, b) \in U \times V \subset W.
	\]
	Thus $W$ is open in the product topology, which completes the proof of \cref{lem-5-5}.
\end{proof}

\begin{proof}[Proof of \cref{lem-5-4}.]
	Let $X_{0} \subset \R^{\infty}$ be a fixed $n$-plane, and let $U \subset \grass{n}{}$ be the set of all $n$-planes $Y$ which project onto $X_{0}$ under the orthogonal projection $\map{p}{\R^{\infty}}{ X_{0}}$. This set $U$ is open since, for each finite $k$, the intersection
	\[
	U_{k}=U \cap \grass{n}{n+k}
	\]
	is known to be an open set. Defining 
	\[ \map{h}{U \times X_{0}}{\pi^{-1} U}\]
	as in \cref{lem-5-2}, it follows from \cref{lem-5-2} that $h \mid _{U_{k} \times X_{0}}$ is continuous for each $k$. Now \cref{lem-5-5} implies that $h$ itself is continuous.
	
	As before, the identity $h^{-1}(Y, y)=(Y, py)$ implies that $h^{-1}$ is continuous. Thus $h$ is a homeomorphism. This completes the proof that $\gamma^{n}$ is locally trivial.
\end{proof}

The following two theorems assert that this bundle $\gamma^{n}$ over $\grass{n}{}$ is a ``universal'' $\R^{n}$-bundle.
\begin{theorem}\label{thm-5-6}
	 Any $\R^{n}$-bundle $\xi$ over a paracompact base space admits a bundle map $\xi \rightarrow \gamma^{n}$.
\end{theorem}
\begin{definition}\label{def:5-3}
Two bundle maps, $\map{f, g}{\xi}{\gamma^{n}}$ are called \textit{bundle-homotopic} if there exists a one-parameter family of bundle maps
\[
\map{h_{t}}{\xi}{\gamma^{n}}\quad 0 \leq t \leq 1,
\]
with $h_{0}=f$, $h_{1}=g$, such that $h$ is continuous as a function of both variables. In other words the associated function
\[\map{h}{E(\xi) \times[0,1]}{E(\gamma^{n})}
\]
must be continuous.	
\end{definition}


\begin{theorem}\label{thm-5-7}
	Any two bundle maps from an $\R^{n}$-bundle to $\gamma^{n}$ are bundle-homotopic.
\end{theorem}

\subsection*{Paracompact Spaces}

Before beginning the proof of \cref{thm-5-6} and \cref{thm-5-7}, let us review the definition and the basic theorems concerning paracompactness. For further information the reader is referred to \cite{41} or \cite{73}.

\begin{definition}\label{def:5-4}
	A topological space $\B $ is \textit{paracompact} if $\B $ is a Hausdorff space and if, for every open covering $\left\{U_{\alpha}\right\}$ of $\B $, there exists an open covering $\left\{V_{\beta}\right\}$ which
	\begin{enumerate}[label=\arabic*),leftmargin=2\parindent ]
		\item is a \textit{refinement} of $\left\{U_{\alpha}\right\}$: that is each $V_{\beta}$ is contained in some $U_{\alpha}$, and
		\item is \textit{locally finite}: that is each point of $\B $ has a neighborhood which intersects only finitely many of the $V_{\beta}$.
	\end{enumerate}
\end{definition}

Nearly all familiar topological spaces are paracompact. For example (see the above references):
\begin{theorem}[A. H. Stone]\label{thm-5-8}
Every metric space is paracompact.	
\end{theorem}
\begin{theorem}[Morita]\label{thm-5-9}
	If a regular topological space is the countable union of compact subsets, then it is paracompact.
\end{theorem}
\begin{corollary}\label{cor-5-9}
	The direct limit of a sequence $K_{1} \subset K_{2} \subset K_{3} \subset \dots$ of compact spaces is paracompact. In particular the infinite Grassmann space $\grass{n}{}$ is paracompact.
\end{corollary}

For it follows from \cite[\S~18.4]{82} that such a direct limit is regular. (The reader should have no difficulty in supplying a proof.)
\begin{theorem}[Dieudonné]\label{thm-5-11}
	Every paracompact space is normal.
\end{theorem}
The proof of \cref{thm-5-6} will be based on the following:
\begin{lemma}\label{lem-5-9}
	For any fiber bundle $\xi$ over a paracompact space $\B $, there exists a locally finite covering of $\B $ by countably many open sets $U_{1}, U_{2}, U_{3}, \dots$, so that $\xi \mid_{ U_{i}}$ is trivial for each $ i $.
\end{lemma} 
\begin{proof}
	Choose a locally finite open covering $\left\{V_{\alpha}\right\}$ so that each $\xi \mid _{V_{a}}$ is trivial; and choose an open covering $\left\{W_{a}\right\}$ with $\overline{W}_{\alpha} \subset V_{a}$ for each $\alpha$. (Compare \cite[p.~171]{41}.) Let $\map{\lambda_{\alpha}}{\B }{\R} $ be a continuous function which takes the value $1$ on $\overline{W}_{\alpha}$ and the value $0$ outside of $V_{\alpha} $. For each non-vacuous finite subset $S$ of the index set $\{\alpha\}$, let $U(S) \subset \B $ denote the set of all $b \in B$ for which
	\[
	\min_{\alpha \in S} \lambda_{\alpha}(b)>\max_{\alpha \notin S} \lambda_{\alpha}(b).
	\]
	Let $U_{k}$ be the union of those sets $U(S)$ for which $S$ has precisely $k$ elements.
	
	Clearly each $U_{k}$ is an open set, and
	\[
	\B =U_{1}\cup U_{2} \cup U_{3} \cup \dots\ .
	\]
	For, given $b \in \B $, if precisely $k$ of the numbers $\lambda_{a}(b)$  are positive, then $b \in U_{k} $. If $a$ is any element of the set $S$, note that
	\[
	U(S) \subset V_{\alpha}.
	\]
	Since the covering $\left\{V_{\alpha}\right\}$ is locally finite, it follows that $\left\{U_{k}\right\}$ is locally finite. Furthermore, since each $\xi\mid _{V_{\alpha}}$ is trivial, it follows that each. $\xi \mid _{U(S)}$ is trivial. But the set $U_{k}$ is equal to the dis.joint union of its open subsets $U(S) $. Therefore $\xi \mid _{U_{k}}$ is also trivial.
\end{proof}

The bundle map $\map{f}{\xi}{\gamma^{n}}$ can now be constructed just as in the proof of \cref{lem-5-3}. Details will be left to the reader. This proves \cref{thm-5-6}.

\begin{proof}[Proof of \cref{thm-5-6}]
	Any bundle map $\map{f}{\xi}{\gamma^{n}} $ determines a map
\[
	\map{\hat{f}}{E(\xi)}{\R^{\infty}}\]
	whose restriction to each fiber of $\xi$ is linear and injective. Conversely $\hat{f}$ determines $f$ by the identity
	\[
	f(e)=(\hat{f} \text { (fiber through } e), \hat{f}(e))
	\]
	Let $\map{f, g}{\xi}{\gamma^{n}}$ be any two bundle maps.
	
	Case 1. Suppose that the vector $\hat{f}(e) \in \R^{\infty}$ is never equal to a negative multiple of $\hat{g}(e)$ for $e \neq 0$, $e \in E(\xi)$. Then the formula
	\[
	\hat{h}_{t}(e)=(1-t) \hat{f}(e)+\hat{\mathrm{tg}}(e), \quad 0 \leq t \leq 1
	\]
defines a homotopy between $\hat{f}$ and $\hat{g}$. To prove that $\hat{h}$ is continuous as a function of both variables, it is only necessary to prove that the vector space operations in $\R^{\infty}$ (i.e., addition, and multiplication by scalars) are continuous. But this follows easily from \cref{lem-5-5}. Evidently
$\hat{h}_t(e)\neq0$ if $e$ is a non-zero vector of $E(\xi)$. Hence we can define
$\map{h}{E(\xi)\times[0,1]}{E(\eta)}$ by
\[h_t(e)=(\hat{h}_t(\text{fiber through } e),\hat{h}_t(e)).\]
To prove that $h$ is continuous, it is sufficient to prove that the 
corresponding function
\[\map{\overline{h}}{\B (\xi)\times[0,1]}{\grass{n}{}}\]
on the base space is continuous. Let $U$ be an open subset of $\B (\xi)$ with $\xi|_{U}$ trivial, and let $s_1,\dots,s_n$ be nowhere dependent cross-sections of
$\xi|_{U}$. Then $\overline{h}|_{ U \times [0,1]}$ can be considered as the composition of
\begin{enumerate}[label=\arabic*),leftmargin=2\parindent ]
	\item a continuous function $b$, $t\mapsto (\hat{h}_ts_1(b)),\dots, \hat{h}_ts_n(b)))$ from $U \times [0,1]$
	to the ``infinite Stiefel manifold'' $\stiefel{n}{}{\infty} \subset \R^\infty \times \dots \times \R^\infty$, and
	\item the canonical projection $\map{q}{\stiefel{n}{}{\infty}}{\grass{n}{}}$.
\end{enumerate}
 Using \cref{lem-5-5} it is seen that $q$ is continuous. Therefore $\overline{h}$ is continuous;
hence the bundle-homotopy $h$ between $f$ and $g$ is continuous.

General Case. Let $\map{f, g}{\xi}{\gamma^n}$ be arbitrary bundle maps. A bundle
map
\[\map{d_1}{\gamma^n}{\gamma^n}\]
is induced by the linear transformation $\R^\infty\to \R^\infty$ which carries the $i$-th
basis vector of $\R^\infty$ to the $(2i-1)$-th. Similarly $\map{d_2}{\gamma^n}{\gamma^n}$ is induced
by the linear transformation which carries the $i$-th basis vector to the
$2i$-th. Now note that three bundle-homotopies
\[f\sim d_1\circ f\sim d_2\circ g\sim g\]
are given by three applications of Case $1$. Hence $f\sim
g$. 
\end{proof} 


\subsection*{Characteristic Classes of Real $n$-Plane Bundles}
Using \cref{thm-5-6,thm-5-7}, it is possible to give a precise definition of the
concept of characteristic class. First observe the following.
\begin{corollary}\label{cor-5-10}
	Any $\R^n$-bundle $\xi$ over a paracompact space
	$\B $ determines a unique homotopy class of maps
	\[\map{\bar{f}_\xi}{\B }{\grass{n}{}}.\]
\end{corollary}
\begin{proof}
	Let $\map{f_\xi}{\xi}{\gamma^n}$ be any bundle map, and let $\bar{f}_\xi$ be the induced
	map of base spaces.
\end{proof}
Now let $\Lambda$ be a coefficient group or ring and let
\[c\in\homology^i(\grass{n}{};\Lambda)\]
be any cohomology class. Then $\xi$ and $c$ together determine a cohomology
class
\[{\bar{f}_\xi}^{*} c\in\homology^i(\B ;\Lambda)\]
This class will be denoted briefly by $c(\xi)$.







\begin{definition}\label{def:5-6}
	$c(\xi)$ is called the \textit{characteristic cohomology class} of
	$\xi$ determined by $c$.
	
\end{definition}Note that the correspondence $\xi \mapsto(\xi)$ is natural with respect to
bundle maps. (Compare Axiom 2 in \S~4). Conversely, given any 
correspondence
\[\xi\longmapsto c(\xi)\in \homology^i\left(\B (\xi);\Lambda\right)\]
which is natural with respect to bundle maps, we have
\[c(\xi)={\bar{f}_\xi}^{*}c(\gamma^n).\]

Thus the above construction is the most general one. Briefly speaking:
The ring consisting of all characteristic cohomology classes for $\R^n$-bundles over paracompact base spaces with coefficient ring $\Lambda$ is canonically isomorphic to the cohomology ring $\homology^* (\grass{n}{}; \Lambda)$.

These constructions emphasize the importance of computing the
cohomology of the space $\grass{n}{}$. The next two sections will give one 
procedure for computing this cohomology, at least modulo $2$.


\begin{remark*}
Using the ``covering homotopy theorem'' (compare 
\cite{84,83}.), \cref{cor-5-10} can be sharpened as follows: Two $\R^n$-bundles $\xi$ and $\eta$ over the paracompact space $\B $ are isomorphic if and
only if the mapping $\overline{f}_\xi$ of \cref{cor-5-10} is homotopic to $\overline{f}_\eta$.	
\end{remark*}

\section*{Problems}
Here are five problems for the reader.
\begin{problem}\label{prob-5-A}
	Show that the Grassmann manifold $\grass{n}{n+k}$ can be
	made into a smooth manifold as follows: a function $\map{f}{\grass{n}{n+k}}{\R}$
	belongs to the collection $F$ of smooth real valued functions if and only if
	$\map{f\circ q}{\stiefel{n}{}{n+k}}{\R}$ is smooth.
\end{problem}
\begin{problem}\label{prob-5-B}
	Show that the tangent bundle of $\grass{n}{n+k}$ is 
	isomorphic to $\Hom(\gamma^n(\R^{n+k} ),\gamma^\perp)$; where $ \gamma^\perp $ denotes the orthogonal 
	complement of $\gamma^n(\R^{n+k} )$ in $\mathcal{E}^{n+k}$. Now consider a smooth manifold $M\subset \R^{n+k}$.
	If $\map{\bar{g}}{M}{\grass{n}{n+k}}$ denotes the generalized Gauss map, show that
	\[\map{\mathsf{D}\bar{g}}{\mathsf{D} M}{\mathsf{D}\grass{n}{n+k}}\]
	gives rise to a cross-section of the bundle
	\[\Hom(\tau_M,\Hom(\tau_M,\nu))\cong \Hom(\tau_M\otimes\tau_M,\nu).\]
	(This cross-section is called the ``\textit{second fundamental form}'' of $M$.)
\end{problem}

\begin{problem}\label{prob-5-C}
	Show that $\grass{n}{m}$ is diffeomorphic to the smooth 
	manifold consisting of all $m \times m$ symmetric, idempotent matrices of trace $n$.
	Alternatively show that the map
	\[(x_1,\dots,x_n)\longmapsto x_1\wedge\dots\wedge x_n\]
	from $\stiefel{n}{}{m}$ to the exterior power $\Lambda^n(\R^m)$ gives rise to a smooth 
	embedding of $\grass{n}{m}$ in the projective space $\mathsf{Gr}_1(\Lambda^n(\R^m))\cong\rp{\binom{m}{n}-1}$. 
	(Compare \cite[\S~7]{81}.)
\end{problem}
\begin{problem}\label{prob-5-D}
	Show that $\grass{n}{n+k}$ has the following symmetry 
	property. Given any two $n$-planes $X, Y \subset \R^n$ there exists an orthogonal
	automorphism of $\R^{n+k}$ which interchanges $X$ and $Y$. Whitehead \cite{82}
	defines the angle $\alpha(X, Y)$ between $n$-planes as the maximum over all unit
	vectors $x\in X$ of the angle between $x$ and $Y$. Show that $\alpha$ is a metric
	for the topological space $\grass{n}{n+k}$ and show that
	\[\alpha(X, Y)=\alpha(X^\perp, Y^\perp).\]
\end{problem}
\begin{problem}\label{prob-5-E}
	Let $\xi$ be an $\R^n$-bundle over $\B $.
	\begin{enumerate}[label=\arabic*),leftmargin=2\parindent ]
		\item Show that there exists a vector bundle $\eta$ over $\B $ with $\xi\oplus\eta$
		trivial if and only if there exists a bundle map
		\[\xi\longrightarrow\gamma^n(\R^{n+k})\]
		for large $k$. If such a map exists, $\xi$ will be called a \textit{bundle of
		finite type}.
		\item Now assume that $\B $ is normal. Show that $\xi$ has finite type if
		and only if $\B $ is covered by finitely many open sets $U_1,\dots,
		U_r$
		with $\xi|_{U_i}$ trivial.
		\item If $\B $ is paracompact and has finite covering dimension, show
		(using the argument of \cref{lem-5-9}) that every $\xi$ over $\B $ has finite type.
		\item Using Stiefel-Whitney classes, show that the vector bundle $\gamma^1$
		over $\rp{\infty}$ does not have finite type.
	\end{enumerate}
\end{problem}