\chapter{The Cohomology Ring $\mathsf{H}^{*}(\mathsf{Gr}_{n}; \mathbb{Z} / 2)$}
Still assuming the existence of Stiefel-Whitney classes, this section will compute the mod $2$ cohomology of the infinite Grassmann manifold $\grass{n}{}=\grass{n}{\infty}$, and will also prove a uniqueness theorem for Stiefel-Whitney classes. Recall that the canonical $n$-plane bundle over $\grass{n}{}$ is denoted by $\gamma^{n}$.\linebreak
\begin{theorem}\label{thm-7-1}
	 The cohomology ring $\homology^{*}(\grass{n}{} ; \Z / 2)$ is a polynomial algebra over $\Z / 2$ freely generated by the Stiefel-Whitney classes $w_{1}(\gamma^{n}), \dots, w_{n}(\gamma^{n})$.
\end{theorem}
\vspace{.3cm}
To prove this result, we first show the following.
\vspace{.3cm}
\begin{lemma}\label{lem-7-2}
	There are no polynomial relations among the $w_{i}(\gamma^{n})$.
\end{lemma}
\begin{proof}
	Suppose that there is a relation of the form $p(w_{1}(\gamma^{n}), \dots,w_{n}(\gamma^{n}))=0$, where $p$ is a polynomial in $n$ variables with mod 2 coefficients. By \cref{thm-5-6}, for any $n$-plane bundle $\xi$ over a paracompact base space there exists a bundle map $\map{g}{\xi}{\gamma^{n}}$. Hence
	\[
	w_{i}(\xi)=\overline{g}^{*}(w_{i}(\gamma^{n}))
	\]
	where $\overline{g}$ is the map of base spaces induced by $g$. It follows that the cohomology classes $w_{i}(\xi)$ must satisfy the corresponding relation
	\[
	p(w_{1}(\xi), \dots, w_{n}(\xi))=\overline{g}^{*} p(w_{1}(\gamma^{n}), \dots, w_{n}(\gamma^{n}))=0.
	\]
	Thus to prove \cref{lem-7-2} it will suffice to find some $n$-plane bundle $\xi$ so that there are no polynomial relations among the classes $w_{1}(\xi), \dots, w_{n}(\xi)$ Consider the canonical line bundle $\gamma^{1}$ over the infinite projective space $\rp{\infty} $. Recall from $\S 4.3$ that $\homology^{*}(\rp{\infty} ; \Z / 2)$ is a polynomial algebra over $\Z / 2$ with a single generator a of dimension $1$ , and recall that $w(\gamma^{1})=1+a$. Forming the $n$-fold Cartesian product $X=\rp{\infty} \times \dots \times \rp{\infty}$ it follows that $\homology^{*}(X ; \Z / 2)$ is a polynomial algebra on $n$ generators $a_{1}, \dots, a_{n}$ of dimension $1 $. (Compare \Cref{app-A}, \cref{thm-A-6}; or \cite[p.~247]{57}.) Here  $a_{i}$ can be defined as the image $\pi_{i}^{*}(a)$ induced by the projection map $\map{\pi_{i}}{X}{\rp{\infty}}$ to the $i$-th factor. Let $\xi$ be the $n$-fold cartesian product
	\[
	\xi=\gamma^{1} \times \dots \times \gamma^{1} \cong(\pi_{1}^{*} \gamma^{1}) \oplus \dots \oplus(\pi_{n}^{*} \gamma^{1}).
	\]
	Then $\xi$ is an $n$-plane bundle over $X=\rp{\infty} \times \dots \times \rp{\infty}$, and the total Stiefel-Whitney class
	\[
	w(\xi)=w(\gamma^{1}) \times \dots \times w(\gamma^{1})=\pi_{1}^{*}(w(\gamma^{1})) \dots \pi_{n}^{*}(w(\gamma^{1}))
	\]
	is equal to the $n$-fold product
	\[
	(1+a) \times \dots \times(1+a)=(1+a_{1})(1+a_{2}) \dots(1+a_{n})
	\]
	In other words
\begin{align*}
		&w_{1}(\xi)=a_{1}+a_{2}+\dots+a_{n} \\
		&w_{2}(\xi)=a_{1} a_{2}+a_{1} a_{3}+\dots+a_{1} a_{n}+\dots+a_{n-1} a_{n} \\
		&w_{n}(\xi)=a_{1} a_{2} \dots a_{n}
	\end{align*}
	and in general $w_{k}(\xi)$ is the $k$-th \textit{elementary symmetric function} of $a_{1}, \dots, a_{n} $. It is proved in textbooks on algebra, that the $n$ elementary symmetric functions in $n$ indeterminates over a field do not satisfy any polynomial relations. (See for example  \cite[pp.~132-134]{76} or \cite[pp.~79,176]{75}.) Thus the classes $w_{1}(\xi), \dots, w_{n}(\xi)$ are algebraically independent over $\Z / 2$, and it follows as indicated above that $w_{1}(\gamma^{n}), \dots, w_{n}(\gamma^{n})$ are also algebraically independent.
\end{proof}\vspace{-.8cm}
\begin{proof}[Proof of \cref{thm-7-1}]
	We have shown that $\homology^{*}(\grass{n}{})$, with mod 2 coefficients, contains a polynomial algebra over $\Z / 2$ freely generated by $w_{1}(\gamma^{n}), \dots,$ $w_{n}(\gamma^{n})$. Using a counting argument, we will show that this sub-algebra actually coincides with $\homology^{*}(\grass{n}{})$.
	
	Recall from \cref{def:6-7} that the number of $r$-cells in the CW-complex $\grass{n}{}$ is equal to the number of partitions of $r$ into at most $n$ integers. Hence the rank of $\homology^{r}(\grass{n}{})$ over $\Z / 2$ is at most equal to this number of partitions. (In fact, if $C^{r}$ denotes the group of mod $2$ $r$-cochains for this CW-complex, and if $Z^{r} \supset B^{r}$ denote the corresponding cocycle and coboundary groups, then the number of $r$-cells equals
	\[
	\rank(C^{r}) \geq \rank(Z^{r}) \geq \rank(Z^{r} / B^{r})=\rank(\homology^{r}) \text{.)}
	\]
	On the other hand the number of distinct monomials of the form
	\[w_{1}(\gamma^{n})^{r_{1}} \dots w_{n}(\gamma^{n})^{r_n}\]
	in $\homology^{r}(\grass{n}{})$ is also precisely equal to the number of partitions of $r$ into at most $n$ integers. For to each sequence $r_{1}, \dots, r_{n}$ of non-negative integers with
	\[
	r_{1}+2 r_{2}+\dots+nr_{n}=r
	\]
	we can associate the partition of $r$ which is obtained from the $n$-tuple
	\[
	r_{n}, r_{n}+r_{n-1}, \dots, r_{n}+r_{n-1}+\dots+r_{1}
	\]
	by deleting any zeros which may occur; and conversely.
	
	Since these monomials are known to be linearly independent mod $2$, it follows that the inequalities above must all actually be equalities: The module $\homology^{r}(\grass{n}{})$ over $\Z / 2$ has rank equal to the number of partitions of $r$ into at most $n$ integers, and has a basis consisting of the various monomials $w_{1}(\gamma^{n})^{r_{1}} \dots w_{n}(\gamma^{n})^{r_{n}}$ of total dimension $r$.
\end{proof} 

It follows incidentally that the natural homomorphism $\map{\overline{g}^{*}}{\homology^{*}(\grass{n}{})}{\homology^{*}(\rp{\infty} \times \dots \times \rp{\infty})}$ maps $\homology^{*}(\grass{n}{})$ isomorphically onto the algebra consisting of all polynomials in the indeterminates $a_{1}, \dots, a_{n}$ which are invariant under all permutations of these $n$ indeterminates.

\subsection*{Uniqueness of Stiefel-Whitney Classes}
At this point we have not yet shown that there exist Stiefel-Whitney classes $w_{i}(\xi)$ satisfying the four axioms of \Cref{ch-4}. Before proving existence, we will prove the following.


\begin{theorem}[UNIQUENESS THEOREM]\label{thm-7-3}
	There exists at most one correspondence $\xi \mapsto w(\xi)$ which assigns to each vector bundle over a paracompact base space a sequence of cohomology classes satisfying the four axioms for Stiefel-Whitney classes.
\end{theorem}
\begin{proof}
	Suppose that there were two such, say $\xi \mapsto w(\xi)$ and $\xi \mapsto \widetilde{w}(\xi)$ For the canonical line bundle $\gamma_{1}^{1}$ over $\rp{1}$ we have
	\[
	w(\gamma_{1}^{1})=\widetilde{w}(\gamma_{1}^{1})=1+a
	\]
	by Axioms 1 and 4. Embedding $\gamma_{1}^{1}$ in the line bundle $\gamma^{1}$ over the infinite projective space $\rp{\infty}$, it follows that
	\[
	w(\gamma^{1})=\widetilde{w}(\gamma^{1})=1+a
	\]
	by Axioms 1 and $2 $. Passing to the $n$-fold cartesian product
	\[
	\xi=\gamma^{1} \times \dots \times \gamma^{1} \cong \pi_{1}^{*} \gamma^{1} \oplus \dots \oplus \pi_{n}^{*} \gamma^{1}
	\]
	it follows that
	\[
	w(\xi)=\widetilde{w}(\xi)=(1+a_{1}) \dots(1+a_{n})
	\]
	by Axioms 2 and 3. Now using the existence of a bundle map $\xi \rightarrow \gamma^{n}$, and the fact that $\homology^{*}(\grass{n}{})$ injects monomorphically into $\homology^{*}(\rp{\infty} \times \dots \times \rp{\infty})$ it follows that $w(\gamma^{n})=\widetilde{w}(\gamma^{n})$
	
	For any $n$-plane bundle $\eta$ over a paracompact base space, choosing a bundle map $\map{f}{\eta}{\gamma^{n}}$, it follows immediately that
	\[
	w(\eta)=\overline{f}^{*} w(\gamma^{n})=\overline{f}^{*} \widetilde{w}(\gamma^{n})=\widetilde{w}(\eta).\qedhere
	\]
\end{proof}
\begin{remark*}
	Using essentially this same argument, it would not be difficult to prove a corresponding uniqueness theorem for Stiefel-Whitney classes, working in the much smaller category consisting of smooth vector bundles and smooth bundle mappings, all of the base spaces being smooth paracompact manifolds. It would be much more difficult, however, to prove such a result using only tangent bundles of manifolds. Compare \cite{78}.
\end{remark*}


Here are three problems for the reader. The first two are based on \cref{prob-6-c}.

\begin{problem}\label{prob-7-A}
	Identify explicitly the cocycle in $\mathrm{C}^{r}(\grass{n}{}) \cong \homology^{r}(\grass{n}{})$ which corresponds to the Stiefel-Whitney class $w_{r}(\gamma^{n})$.
\end{problem}
\begin{problem}\label{prob-7-B}
	Show that the cohomology algebra $\homology^{*}(\grass{n}{n+k})$ over $\Z / 2$ is generated by the Stiefel-Whitney classes $w_{1}, \dots, w_{n}$ of $y^{n}$ and the dual classes $\overline{w}_{1}, \dots, \overline{w}_{k}$, subject only to the $n+k$ defining relations
	\[
	(1+w_{1}+\dots+w_{n})(1+\bar{w}_{1}+\dots+\bar{w}_{k})=1.
	\]
	(Reference: \cite[p.~190]{79}.)
\end{problem}
\begin{problem}\label{prob-7-C}
	Let $\xi^{m}$ and $\eta^{n}$ be vector bundles over a paracompact base space. Show that the Stiefel-Whitney classes of the tensor product $\xi^{m} \otimes \eta^{n}$ (or of the isomorphic bundle $\Hom(\xi^{m}, \eta^{n})$ ) can be computed as follows. If the fiber dimensions $m$ and $n$ are both $ 1 $ , then
	\[
	w_{1}(\xi^{1} \otimes \eta^{1})=w_{1}(\xi^{1})+w_{1}(\eta^{1})
	\]
	More generally there is a universal formula of the form
	\[
	w(\xi^{m} \otimes \eta^{n})=p_{m, n}\left(w_{1}(\xi^{m}), \dots, w_{m}(\xi^{m}), w_{1}(\eta^{n}), \dots, w_{n}(\eta^{n})\right)
	\]
	where the polynomial $p_{m, n}$ in $m+n$ variables can be characterized as follows. If $\sigma_{1}, \dots, \sigma_{m}$ are the elementary symmetric functions of indeterminates $t_{1}, \dots, t_{m}$, and if $\sigma_{1}^{\prime}, \dots, \sigma_{n}^{\prime}$ are the elementary symmetric functions of $t_{1}^{\prime}, \dots, t_{n}^{\prime}$, then
	\[
	p_{m, n}(\sigma_{1}, \dots, \sigma_{m}, \sigma_{1}^{\prime}, \dots, \sigma_{n}^{\prime})=\prod_{i=1}^{m} \prod_{j=1}^{n}(1+t_{i}+t_{j}^{\prime}).
	\]
	[Hint: The cohomology of $\grass{m}{} \times \grass{n}{}$ can be computed by the Künneth Theorem (\cref{thm-A-6}). The formula for $w(\xi^{m} \otimes \eta^{n})$ can be verified first in the special case when $\xi^{m}$ and $\eta^{n}$ are Whitney sums of line bundles.]
\end{problem}

