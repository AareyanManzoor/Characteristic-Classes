\chapter{Existence of Stiefel-Whitney Classes}\label{ch-8}
We now proceed to prove the existence of Stiefel-Whitney classes by giving a construction in terms of known operations. For any n-plane bundle $\xi$ with total space $E$, base space $\B $ and projection map $\pi$, we denote by $E_{0}$ the set of all non-zero elements of $E$, and by $F_{0}$ the set of all non-zero elements of a typical fiber $F=\pi^{-1}(b)$. Clearly $F_{0}= F \cap E_{0}$.

Using singular theory and one of several techniques (e.g. spectral sequences or that of \Cref{ch-10}) we have that
\[\homology^{i}(F, F_{0} ; \Z / 2)=\begin{cases}
	0 &\text { for } i \neq  n  \\
	\Z / 2 &\text { for } i= n 
	\end{cases}
\]
and that
\[\homology^{i}(E, E_{0} ; \Z / 2)\cong\begin{cases}
	0& \text { for } i< n  \\
	\homology^{i- n }(\B  ; \Z / 2)& \text { for } i \geq  n  
\end{cases}
\]
(This can be seen intuitively, though not rigorously, as follows: The unit $ n $-cell is a deformation retract of $\R^{ n }$ and the unit $(n -1)$-sphere is a deformation retract of $(\R^{ n }\setminus\{0\})=\R_{0}^{ n } $. For $\B $ paracompact, we know that we can put a Euclidean metric on $E$. Then the subset $E^{\prime}$ consisting of all vectors $x \in E$ with $x \boldsymbol{\cdot} x \leq 1$ is evidently a deformation retract of $E$. Similarly the set $E^{\prime \prime}$ consisting of vectors $x \in E$ with $x \cdot x=1$ is a deformation retract of $E_{0} $. Hence $\homology^{*}(E^{\prime}, E^{\prime \prime}) \cong \homology^{*}(E, E_{0})$. Now suppose that $\B $ is a cell complex, with a fine enough cell subdivision so that the restriction of $\xi$ to each cell $c^{k}$ is a trivial bundle. Then the inverse image of the $k$-cell $ c^{k}$ in $E^{\prime}$ is a product cell of dimension $ n +k$. Thus $E^{\prime}$ can be obtained from the subset $E^{\prime \prime}$ by adjoining cells of dimension $\geq  n $, one $( n +k)-$cell corresponding to each $k$-cell of $\B $. It follows that $\homology^{i}(E^{\prime}, E^{\prime \prime})=0$ for $i< n $. With a little faith, it follows also that $\homology^{ n +k}(E^{\prime}, E^{\prime \prime}) \cong \homology^{k}(\B )$.)

Rigorously and more explicitly, the following statement will be proved in \cref{ch-10}. The coefficient group $\Z / 2$ is to be understood.

\begin{theorem}\label{thm-8-1}
	The group $\homology^{i}(E, E_{0})$ is zero for $i< n $, and $\homology^{ n }(E, E_{0})$ contains a unique class $u$ such that for each fiber $F=\pi^{-1}(b)$ the restriction
	\[
	u \mid _{(F, F_{0})} \in \homology^{ n }(F, F_{0})
	\]
	is the unique non-zero class in $\homology^{ n }(F, F_{0})$. Furthermore the correspondence $x \mapsto x \smile u$ defines an isomorphism $\homology^{k}(E) \rightarrow \homology^{k+ n }(E, E_{0})$ for every $k$. (We call $u$ the fundamental cohomology class.)
\end{theorem}

On the other hand the projection $\map{\pi}{E}{\B }$ certainly induces an isomorphism $\homology^{k}(\B ) \rightarrow \homology^{k}(E)$, since the zero cross-section embeds $\B $ as a deformation retract of $E$ with retraction mapping $\pi$.

\begin{definition}\label{def:8-2}
	The Thom isomorphism $\map{\varphi}{\homology^{k}(\B )}{\homology^{k+ n }(E, E_{0})}$ is defined to be the composition of the two isomorphisms
	\[
	\begin{tikzcd}
		\homology^{k}(\B ) \arrow[r, "\pi^{*}"] & \homology^{k}(E) \arrow[r, "\cup u"] & \homology^{k+ n }(E, E_{0})
	\end{tikzcd}.
	\]
\end{definition}

Next we will make use of the Steenrod squaring operations in $\homology^{*}(E, E_{0})$. These operations can be characterized by four basic properties, as follows. (Compare \cite{80}.) Again mod $2$ coefficients are to be understood.

\begin{enumerate}[label=(\arabic*),leftmargin=2\parindent ]
	\item For each pair $X \supset Y$ of spaces and each pair $ n , i$ of non-negative integers there is defined an additive homomorphism
	\[
	\map{\sq^{i}}{\homology^{ n }(X, Y)}{\homology^{ n +i}(X, Y)}.
	\]
	(This homomorphism is called ``square upper $i$.'')
	\item \textsc{Naturality.} If $\map{f}{(X,Y)}{(X',Y')}$ then $\sq^{i} \circ f^{*}=f^{*} \circ \sq^{i}$.
	\item If $a \in \homology^{ n }(X, Y)$, then $\sq^{0}(a)=a$, $\sq^{ n }(a)=a \smile a$, and $\sq^{i}(a)=0$ for $i> n $. (Thus the most interesting squaring operations are those for which $0<i< n $.)
	\item \textsc{The Cartan formula.} The identity
	\[
	\sq^{k}(a \smile b)=\sum_{i+j=k} \sq^{i}(a) \smile \sq^{j}(b)
	\]
	is valid whenever $a \smile b$ is defined.
\end{enumerate}
 Using these squaring operations together with the Thom isomorphism $\varphi$, the \textit{Stiefel-Whitney class} $w_{i}(\xi) \in \homology^{i}(\B )$ can now be defined by Thom's identity
\[
w_{i}(\xi)=\varphi^{-1} \sq^{i} \varphi(1).
\]
In other words $w_{i}(\xi)$ is the unique cohomology class in $\homology^{i}(\B )$ such that $\varphi(w_{i}(\xi))=\pi^{*} w_{i}(\xi) \smile u$ is equal to $\sq^{i} \varphi(1)=\sq^{i}(u)$.

For many purposes it is convenient to introduce the \textit{total squaring operation}
\[
\sq(a)=a+\sq^{1}(a)+\sq^{2}(a)+\dots+\sq^{ n }(a)
\]
for $a \in \homology^{ n }(X, Y)$. Note that the Cartan formula can now be expressed by the equation
\[
\sq(a \smile b)=(\sq( a)) \smile(\sq( b)).
\]
Similarly the corresponding equation for the Steenrod squares of a cross product becomes simply
\[
\sq(a \times b)=(\sq (a)) \times(\sq (b)).
\]
In terms of this total squaring operation, the total Stiefel-Whitney class of a vector bundle is clearly determined by the formula
\[
w(\xi)=\varphi^{-1} \sq \varphi(1)=\varphi^{-1} \sq(u)
\]

\section*{Verification of the Axioms}
With this definition, the four axioms for Stiefel-Whitney classes can be checked as follows.\vspace{.3cm}

\noindent\textsf{AXIOM 1.} Using properties (1) and (3) of the squaring operations, it is clear that $w_{i}(\xi) \in \homology^{i}(\B )$, with $w_{0}(\xi)=1$, and with $w_{i}(\xi)=0$ for $i$ greater than the fiber dimension $ n $.\vspace{.3cm}

\noindent\textsf{AXIOM 2.} Any bundle map $\map{f}{\xi}{\xi^{\prime}}$ clearly induces a map $\map{g}{(E, E_{0})}{(E^{\prime}, E_{0}^{\prime})}$. Furthermore if $u^{\prime}$ denotes the fundamental cohomology class in $\homology^{ n }(E^{\prime}, E_{0}^{\prime})$, then $g^{*}(u^{\prime})$ is equal to the class $u \in \homology^{ n }(E, E_{0})$ by the definition of $u$ (\cref{thm-8-1}). It now follows easily that the Thom isomorphisms $\varphi$ and $\varphi^{\prime}$ satisfy the naturality condition
\[
g^{*} \circ \varphi^{\prime}=\varphi \circ \bar{f^{*}}.
\]
Hence, using property (2), it follows that
\[
\bar{f^{*}} w_{i}(\xi^{\prime})=w_{i}(\xi),
\]
as required.\vspace{.3cm}

\noindent\textsf{AXIOM 3.} Let us first compute the Stiefel-Whitney classes of a cartesian product $\xi^{\prime \prime}=\xi \times \xi^{\prime}$, with projection map $\map{\pi \times \pi^{\prime}}{E \times E^{\prime}}{\B  \times \B ^{\prime}} $ Consider the fundamental classes
\[
u \in \homology^{m}(E, E_{0}), \quad u^{\prime} \in \homology^{ n }(E^{\prime}, E_{0}^{\prime})
\]
of $\xi$ and $\xi^{\prime}$. Since $E_{0}$ is open in $E$ and $E_{0}^{\prime}$ is open in $E^{\prime}$, the cross product
\[
u \times u^{\prime} \in \homology^{m+ n }(E \times E^{\prime}, E \times E_{0}^{\prime} \cup E_{0} \times E^{\prime})
\]
is defined. (Compare \Cref{app-A}.) Note that the open subset $(E \times E_{0}^{\prime})\cup(E_{0} \times E^{\prime})$ in the total space $E^{\prime \prime}=E \times E^{\prime}$ is precisely equal to the set $E_{0}^{\prime \prime}$ of non-zero vectors in $E^{\prime\prime}$. In fact we claim that $u \times u^{\prime}$ is precisely equal to the fundamental class $u^{\prime \prime} \in \homology^{m+n}(E^{\prime \prime}, E_{0}^{\prime \prime})$. In order to prove this, it suffices to show that the restriction
\[
u \times u^{\prime} \mid_{(F^{\prime \prime}, F_{0}^{\prime \prime})}
\]
is the non-zero cohomology class in $\homology^{m+ n }(F^{\prime \prime}, F_{0}^{\prime \prime})$ for every fiber $F^{\prime \prime}=F \times F^{\prime}$ of $\xi^{\prime \prime}$. But this restriction is evidently equal to the cross product of $u \mid(F, F_{0})$ and $u^{\prime} \mid(F^{\prime}, F_{0}^{\prime})$, and hence is non-zero by \cref{thm-A-5} in the Appendix.

It follows easily that the Thom isomorphisms for $\xi$, $\xi^{\prime}$, and $\xi^{\prime \prime}$ are related by the identity
\[
\varphi^{\prime \prime}(a \times b)=\varphi(a) \times \varphi^{\prime}(b).
\]
In fact if $\bar{a}=\pi^{*}(a) \in \homology^{*}(E)$ and $\bar{b}={\pi^{\prime}}^*(b) \in \homology^{*}(E^{\prime})$, then this follows from the equation
\[
(\bar{a} \times \bar{b}) \smile(u \times u^{\prime})=(\bar{a} \smile u) \times(\bar{b} \smile u^{\prime}),
\]
where there is no sign since we are working modulo $2$.

The total Stiefel-Whitney class of $\xi^{\prime \prime}$ can now be computed by the formula
\[
\varphi^{\prime \prime}(w(\xi^{\prime \prime}))=\sq(u^{\prime \prime})=\sq(u \times u^{\prime})=\sq(u) \times \sq(u^{\prime}).
\]
Setting the right side equal to
\[
\varphi\left(w(\xi)\right) \times \varphi^{\prime}\left(w(\xi^{\prime}))=\varphi^{\prime \prime}(w(\xi) \times w(\xi^{\prime})\right),
\]
and then applying $(\varphi^{\prime \prime})^{-1}$ to both sides, we have proved that
\[
w(\xi \times \xi^{\prime})=w(\xi) \times w(\xi^{\prime}).
\]
Now suppose that $\xi$ and $\xi^{\prime}$ are bundles over a common base space $\B $. Lifting both sides of this equation back to $\B $ by means of the diagonal embedding $\B  \rightarrow \B  \times \B $, we obtain the required formula
\[
w(\xi \oplus \xi^{\prime})=w(\xi) \smile w(\xi^{\prime}).
\]
\noindent\textsf{AXIOM 4.} Let $\gamma_{1}^{1}$ be as usual the twisted line bundle over the circle $\rp{1}$. Then the space of vectors of length $\leq 1$ in the total space $E=E(\gamma_{1}^{1})$ is evidently a Moebius band $M$, bounded by a circle $\stackrel{\bullet}{M}$. Since $M$ is a deformation retract of $E$, and $\stackrel{\bullet}{M}$ a deformation retract of $E_{0}$, we have
\[
\homology^{*}(M, \stackrel{\bullet}{M})^{\prime} \cong \homology^{*}(E, E_{0}).
\]
On the other hand if we embed a $2$-cell $\disk{2}$ in the projective plane $\rp{2}$, then the closure of $\rp{2}\smallsetminus\disk{2}$ is homeomorphic to $M$. Using the Excision Theorem of cohomology theory, it follows that
\[
\homology^{*}(M, \stackrel{\bullet}{M}) \cong \homology^{*}(\rp{2}, \disk{2}).
\]
Hence there are natural isomorphisms
\[
\homology^{i}(E, E_{0}) \longrightarrow \homology^{i}(M, \stackrel{\bullet}{M}) \longleftarrow \homology^{i}(\rp{2}, \disk{2}) \longrightarrow \homology^{i}(\rp{2})
\]
for every dimension $i \neq 0$. The fundamental cohomology class $u \in \homology^{1}(E, E_{0})$ certainly cannot be zero. Hence it must correspond to the generator $a \in \homology^{1}(\rp{2})$ under the composite isomorphism. Hence $\sq^{1}(u)=u \cup u$ must correspond to $\sq^{1}(a)=a \smile a$. But $a \smile a \neq 0$ by \cref{lem-4-3}, so it follows that
\[
w_{1}(\gamma_{1}^{1})=\varphi^{-1} \sq^{1}(u)
\]
must also be non-zero. This concludes the verification of the four axioms.

\subsection*{Problems}
Here are two problems for the reader.
\begin{problem}\label{prob-8-A}
	It follows from \cref{thm-7-1} that the cohomology class $\sq^kw(\xi)$ can be expressed as a polynomial in $w_{1}(\xi), \dots, w_{m+k}(\xi)$. Prove Wu's explicit formula
	\[
	\sq^{k}(w_{m})=w_{k} w_{m}+\binom{k-m}{1} w_{k-1} w_{m+1}+\dots+\binom{k-m}{k}w_{0} w_{m+k}
	\]
	where $\binom{x}{i} =x(x-1) \dots(x-i+1) / i !$, as follows. If the formula is true for $\xi$, show that it is true for $\xi \times \gamma^{1}$. Thus by induction it is true for $\gamma^{1} \times \dots \times \gamma^{1}$, and hence for all $\xi$.
\end{problem}

\begin{problem}\label{prob-8-B}
	If $w(\xi) \neq 1$, show that the smallest $n>0$ with $w_{ n }(\xi) \neq 0$ is a power of 2 . (Use the fact that $\binom{x}{k}$ is odd whenever $x$ is an odd multiple of $k=2^{r}$.)
\end{problem}
